\documentclass[journal,12pt,twocolumn]{IEEEtran}

\usepackage{setspace}
\usepackage{gensymb}

\singlespacing


\usepackage[cmex10]{amsmath}

\usepackage{amsthm}

\usepackage{mathrsfs}
\usepackage{txfonts}
\usepackage{stfloats}
\usepackage{bm}
\usepackage{cite}
\usepackage{cases}
\usepackage{subfig}

\usepackage{longtable}
\usepackage{multirow}

\usepackage{enumitem}
\usepackage{mathtools}
\usepackage{steinmetz}
\usepackage{tikz}
\usepackage{circuitikz}
\usepackage{verbatim}
\usepackage{tfrupee}
\usepackage[breaklinks=true]{hyperref}

\usepackage{tkz-euclide}

\usetikzlibrary{calc,math}
\usepackage{listings}
    \usepackage{color}                                            %%
    \usepackage{array}                                            %%
    \usepackage{longtable}                                        %%
    \usepackage{calc}                                             %%
    \usepackage{multirow}                                         %%
    \usepackage{hhline}                                           %%
    \usepackage{ifthen}                                           %%
    \usepackage{lscape}     
\usepackage{multicol}
\usepackage{chngcntr}

\DeclareMathOperator*{\Res}{Res}

\renewcommand\thesection{\arabic{section}}
\renewcommand\thesubsection{\thesection.\arabic{subsection}}
\renewcommand\thesubsubsection{\thesubsection.\arabic{subsubsection}}

\renewcommand\thesectiondis{\arabic{section}}
\renewcommand\thesubsectiondis{\thesectiondis.\arabic{subsection}}
\renewcommand\thesubsubsectiondis{\thesubsectiondis.\arabic{subsubsection}}


\hyphenation{op-tical net-works semi-conduc-tor}
\def\inputGnumericTable{}                                 %%

\lstset{
%language=C,
frame=single, 
breaklines=true,
columns=fullflexible
}
\begin{document}


\newtheorem{theorem}{Theorem}[section]
\newtheorem{problem}{Problem}
\newtheorem{proposition}{Proposition}[section]
\newtheorem{lemma}{Lemma}[section]
\newtheorem{corollary}[theorem]{Corollary}
\newtheorem{example}{Example}[section]
\newtheorem{definition}[problem]{Definition}

\newcommand{\BEQA}{\begin{eqnarray}}
\newcommand{\EEQA}{\end{eqnarray}}
\newcommand{\define}{\stackrel{\triangle}{=}}
\bibliographystyle{IEEEtran}

\providecommand{\mbf}{\mathbf}
\providecommand{\pr}[1]{\ensuremath{\Pr\left(#1\right)}}
\providecommand{\qfunc}[1]{\ensuremath{Q\left(#1\right)}}
\providecommand{\sbrak}[1]{\ensuremath{{}\left[#1\right]}}
\providecommand{\lsbrak}[1]{\ensuremath{{}\left[#1\right.}}
\providecommand{\rsbrak}[1]{\ensuremath{{}\left.#1\right]}}
\providecommand{\brak}[1]{\ensuremath{\left(#1\right)}}
\providecommand{\lbrak}[1]{\ensuremath{\left(#1\right.}}
\providecommand{\rbrak}[1]{\ensuremath{\left.#1\right)}}
\providecommand{\cbrak}[1]{\ensuremath{\left\{#1\right\}}}
\providecommand{\lcbrak}[1]{\ensuremath{\left\{#1\right.}}
\providecommand{\rcbrak}[1]{\ensuremath{\left.#1\right\}}}
\theoremstyle{remark}
\newtheorem{rem}{Remark}
\newcommand{\sgn}{\mathop{\mathrm{sgn}}}
\providecommand{\abs}[1]{\left\vert#1\right\vert}
\providecommand{\res}[1]{\Res\displaylimits_{#1}} 
\providecommand{\norm}[1]{\left\lVert#1\right\rVert}
%\providecommand{\norm}[1]{\lVert#1\rVert}
\providecommand{\mtx}[1]{\mathbf{#1}}
\providecommand{\mean}[1]{E\left[ #1 \right]}
\providecommand{\fourier}{\overset{\mathcal{F}}{ \rightleftharpoons}}
%\providecommand{\hilbert}{\overset{\mathcal{H}}{ \rightleftharpoons}}
\providecommand{\system}{\overset{\mathcal{H}}{ \longleftrightarrow}}
	%\newcommand{\solution}[2]{\textbf{Solution:}{#1}}
\newcommand{\solution}{\noindent \textbf{Solution: }}
\newcommand{\cosec}{\,\text{cosec}\,}
\providecommand{\dec}[2]{\ensuremath{\overset{#1}{\underset{#2}{\gtrless}}}}
\newcommand{\myvec}[1]{\ensuremath{\begin{pmatrix}#1\end{pmatrix}}}
\newcommand{\mydet}[1]{\ensuremath{\begin{vmatrix}#1\end{vmatrix}}}

\numberwithin{equation}{subsection}

\makeatletter
\@addtoreset{figure}{problem}
\makeatother
\let\StandardTheFigure\thefigure
\let\vec\mathbf

\renewcommand{\thefigure}{\theproblem}

\def\putbox#1#2#3{\makebox[0in][l]{\makebox[#1][l]{}\raisebox{\baselineskip}[0in][0in]{\raisebox{#2}[0in][0in]{#3}}}}
     \def\rightbox#1{\makebox[0in][r]{#1}}
     \def\centbox#1{\makebox[0in]{#1}}
     \def\topbox#1{\raisebox{-\baselineskip}[0in][0in]{#1}}
     \def\midbox#1{\raisebox{-0.5\baselineskip}[0in][0in]{#1}}
\vspace{3cm}
\title{Challenge Problem}
\author{Sri Harsha CH}

\maketitle
\newpage

\bigskip
\renewcommand{\thefigure}{\theenumi}
\renewcommand{\thetable}{\theenumi}

\begin{abstract}
This document explains the concept of finding the determinant of a vandermonde matrix.
\end{abstract}
Download latex-tikz codes from 
%
\begin{lstlisting}
https://github.com/harshachinta/EE5609-Matrix-Theory/tree/master/Challenges/challenge_6
\end{lstlisting}
%
\section{Problem}
Derive an expression for the determinant of a vandermonde matrix.
\begin{align}
    \myvec{1&\alpha_1&\alpha_1^2&\cdots&\alpha_1^{n-1}\\1&\alpha_2&\alpha_2^2&\cdots&\alpha_2^{n-1}\\\vdots&\vdots&\vdots&\vdots&\vdots\\1&\alpha_n&\alpha_n^2&\cdots&\alpha_n^{n-1}}
\end{align}
for real numbers $\alpha_1,\alpha_2,\cdots,\alpha_n$.
\section{Explanation}
For simplification let us consider a $2\times2$ matrix and find the determinant for that.
\begin{align}
    \det{\myvec{1&\alpha_1\\1&\alpha_2}} = \alpha_2 - \alpha_1 = \prod_{1\leq i<j\leq2} (\alpha_{j}-\alpha_{i}) \label{eq:eq1}
\end{align}
From \eqref{eq:eq1}, let us assume that the result is true for $n\geq2$ (inductive step), that is
\begin{align}
    \det{\myvec{1&\alpha_1&\alpha_1^2&\cdots&\alpha_1^{n-1}\\1&\alpha_2&\alpha_2^2&\cdots&\alpha_2^{n-1}\\\vdots&\vdots&\vdots&\vdots&\vdots\\1&\alpha_n&\alpha_n^2&\cdots&\alpha_n^{n-1}}} = \prod_{1\leq i<j\leq n} (\alpha_{j}-\alpha_{i}) \label{eq:eq2}
\end{align}
Let us consider a $(n\times1)\times(n\times1)$ matrix,
\begin{align}
    A = \myvec{1&\alpha_1&\alpha_1^2&\cdots&\alpha_1^{n}\\1&\alpha_2&\alpha_2^2&\cdots&\alpha_2^{n}\\\vdots&\vdots&\vdots&\vdots&\vdots\\1&\alpha_{n+1}&\alpha_{n+1}^2&\cdots&\alpha_{n+1}^{n}}\\
\xleftrightarrow[R_{n-1}\leftarrow R_{n-1}-R_1]{R_n\leftarrow R_n-R_1}
\myvec{1&\alpha_1&\alpha_1^2&\cdots&\alpha_1^{n}\\0&\alpha_2-\alpha_1&\alpha_2^2-\alpha_1^2&\cdots&\alpha_2^{n}-\alpha_1^{n}\\\vdots&\vdots&\vdots&\vdots&\vdots\\0&\alpha_{n+1}-\alpha_1&\alpha_{n+1}^2-\alpha_1^2&\cdots&\alpha_{n+1}^{n}-\alpha_1^{n}}\\
\xleftrightarrow[C_{n-1}\leftarrow C_{n-1}-\alpha_1 C_{n-2}]{C_n\leftarrow C_n-\alpha_1 C_{n-1}}\\
\myvec{1&0&0&\cdots&0\\0&\alpha_2-\alpha_1&(\alpha_2-\alpha_1)\alpha_2&\cdots&(\alpha_2-\alpha_1)\alpha_2^{n-1}\\\vdots&\vdots&\vdots&\vdots&\vdots\\0&\alpha_{n+1}-\alpha_1&(\alpha_{n+1}-\alpha_1)\alpha_{n+1}&\cdots&(\alpha_{n+1}-\alpha_1)\alpha_{n+1}^{n-1}}\\
=\prod_{1< j\leq n+1} (\alpha_{j}-\alpha_{1}) \det{\myvec{1&\alpha_2&\alpha_2^2&\cdots&\alpha_2^{n-1}\\1&\alpha_2&\alpha_2^2&\cdots&\alpha_2^{n-1}\\\vdots&\vdots&\vdots&\vdots&\vdots\\1&\alpha_{n+1}&\alpha_{n+1}^2&\cdots&\alpha_{n+1}^{n-1}}} \label{eq:eq3}
\end{align}
Sub equation \eqref{eq:eq2} in \eqref{eq:eq3} by the inductive hypothesis,
\begin{align}
    \det{A} = \prod_{1< j\leq n+1} (\alpha_{j}-\alpha_{1}) \prod_{2\leq i<j\leq n+1} (\alpha_{j}-\alpha_{i})\\
\implies \det{A} = \prod_{1\leq i<j\leq n+1} (\alpha_{j}-\alpha_{i})
\end{align}
\section{Solution}
\begin{align*}
    \det{\myvec{1&\alpha_1&\alpha_1^2&\cdots&\alpha_1^{n-1}\\1&\alpha_2&\alpha_2^2&\cdots&\alpha_2^{n-1}\\\vdots&\vdots&\vdots&\vdots&\vdots\\1&\alpha_n&\alpha_n^2&\cdots&\alpha_n^{n-1}}} = \prod_{1\leq i<j\leq n} (\alpha_{j}-\alpha_{i})
\end{align*}
\\\end{document}
