\documentclass[journal,12pt,twocolumn]{IEEEtran}

\usepackage{setspace}
\usepackage{gensymb}

\singlespacing


\usepackage[cmex10]{amsmath}

\usepackage{amsthm}

\usepackage{mathrsfs}
\usepackage{txfonts}
\usepackage{stfloats}
\usepackage{bm}
\usepackage{cite}
\usepackage{cases}
\usepackage{subfig}

\usepackage{longtable}
\usepackage{multirow}

\usepackage{enumitem}
\usepackage{mathtools}
\usepackage{steinmetz}
\usepackage{tikz}
\usepackage{circuitikz}
\usepackage{verbatim}
\usepackage{tfrupee}
\usepackage[breaklinks=true]{hyperref}

\usepackage{tkz-euclide}

\usetikzlibrary{calc,math}
\usepackage{listings}
    \usepackage{color}                                            %%
    \usepackage{array}                                            %%
    \usepackage{longtable}                                        %%
    \usepackage{calc}                                             %%
    \usepackage{multirow}                                         %%
    \usepackage{hhline}                                           %%
    \usepackage{ifthen}                                           %%
    \usepackage{lscape}     
\usepackage{multicol}
\usepackage{chngcntr}

\DeclareMathOperator*{\Res}{Res}

\renewcommand\thesection{\arabic{section}}
\renewcommand\thesubsection{\thesection.\arabic{subsection}}
\renewcommand\thesubsubsection{\thesubsection.\arabic{subsubsection}}

\renewcommand\thesectiondis{\arabic{section}}
\renewcommand\thesubsectiondis{\thesectiondis.\arabic{subsection}}
\renewcommand\thesubsubsectiondis{\thesubsectiondis.\arabic{subsubsection}}


\hyphenation{op-tical net-works semi-conduc-tor}
\def\inputGnumericTable{}                                 %%

\lstset{
%language=C,
frame=single, 
breaklines=true,
columns=fullflexible
}
\begin{document}


\newtheorem{theorem}{Theorem}[section]
\newtheorem{problem}{Problem}
\newtheorem{proposition}{Proposition}[section]
\newtheorem{lemma}{Lemma}[section]
\newtheorem{corollary}[theorem]{Corollary}
\newtheorem{example}{Example}[section]
\newtheorem{definition}[problem]{Definition}

\newcommand{\BEQA}{\begin{eqnarray}}
\newcommand{\EEQA}{\end{eqnarray}}
\newcommand{\define}{\stackrel{\triangle}{=}}
\bibliographystyle{IEEEtran}

\providecommand{\mbf}{\mathbf}
\providecommand{\pr}[1]{\ensuremath{\Pr\left(#1\right)}}
\providecommand{\qfunc}[1]{\ensuremath{Q\left(#1\right)}}
\providecommand{\sbrak}[1]{\ensuremath{{}\left[#1\right]}}
\providecommand{\lsbrak}[1]{\ensuremath{{}\left[#1\right.}}
\providecommand{\rsbrak}[1]{\ensuremath{{}\left.#1\right]}}
\providecommand{\brak}[1]{\ensuremath{\left(#1\right)}}
\providecommand{\lbrak}[1]{\ensuremath{\left(#1\right.}}
\providecommand{\rbrak}[1]{\ensuremath{\left.#1\right)}}
\providecommand{\cbrak}[1]{\ensuremath{\left\{#1\right\}}}
\providecommand{\lcbrak}[1]{\ensuremath{\left\{#1\right.}}
\providecommand{\rcbrak}[1]{\ensuremath{\left.#1\right\}}}
\theoremstyle{remark}
\newtheorem{rem}{Remark}
\newcommand{\sgn}{\mathop{\mathrm{sgn}}}
\providecommand{\abs}[1]{\left\vert#1\right\vert}
\providecommand{\res}[1]{\Res\displaylimits_{#1}} 
\providecommand{\norm}[1]{\left\lVert#1\right\rVert}
%\providecommand{\norm}[1]{\lVert#1\rVert}
\providecommand{\mtx}[1]{\mathbf{#1}}
\providecommand{\mean}[1]{E\left[ #1 \right]}
\providecommand{\fourier}{\overset{\mathcal{F}}{ \rightleftharpoons}}
%\providecommand{\hilbert}{\overset{\mathcal{H}}{ \rightleftharpoons}}
\providecommand{\system}{\overset{\mathcal{H}}{ \longleftrightarrow}}
	%\newcommand{\solution}[2]{\textbf{Solution:}{#1}}
\newcommand{\solution}{\noindent \textbf{Solution: }}
\newcommand{\cosec}{\,\text{cosec}\,}
\providecommand{\dec}[2]{\ensuremath{\overset{#1}{\underset{#2}{\gtrless}}}}
\newcommand{\myvec}[1]{\ensuremath{\begin{pmatrix}#1\end{pmatrix}}}
\newcommand{\mydet}[1]{\ensuremath{\begin{vmatrix}#1\end{vmatrix}}}

\numberwithin{equation}{subsection}

\makeatletter
\@addtoreset{figure}{problem}
\makeatother
\let\StandardTheFigure\thefigure
\let\vec\mathbf

\renewcommand{\thefigure}{\theproblem}

\def\putbox#1#2#3{\makebox[0in][l]{\makebox[#1][l]{}\raisebox{\baselineskip}[0in][0in]{\raisebox{#2}[0in][0in]{#3}}}}
     \def\rightbox#1{\makebox[0in][r]{#1}}
     \def\centbox#1{\makebox[0in]{#1}}
     \def\topbox#1{\raisebox{-\baselineskip}[0in][0in]{#1}}
     \def\midbox#1{\raisebox{-0.5\baselineskip}[0in][0in]{#1}}
\vspace{3cm}
\title{Challenge Problem}
\author{Sri Harsha CH}

\maketitle
\newpage

\bigskip
\renewcommand{\thefigure}{\theenumi}
\renewcommand{\thetable}{\theenumi}

\begin{abstract}
This document explains the property of convolution.
\end{abstract}

Download latex-tikz codes from 
%
\begin{lstlisting}
https://github.com/harshachinta/EE5609-Matrix-Theory/tree/master/Challenges/challenge_4
\end{lstlisting}
%
\section{problem}
If 
\begin{align*}
    x_1(n) \ast h_1(n) = y(n)\\
    x_2(n) \ast h_2(n) = y(n)
\end{align*}
then is $h_1(n)=h_2(n)$?\\
\section{Solution}
A finite-length discrete-time signal is basically a sequence, say, $\brak{x_0,…,x_{m-1}}$ which can be written as an m-length vector $\vec{x}\in R^m$.\par
Given two signals $\brak{x_0,…,x_{n-1}}$ and $\brak{h_0,…,h_{m-1}}$, the (linear) convolution of the two is a $m+n-1$ length signal  defined as
\begin{align}
    y\brak{t}=\brak{h\ast x}_t=\sum_{\tau=0}^{\tau=n-1} x_{\tau} h_\brak{t-\tau}\label{eq:1}\\
    0\leq t < m+n-1\notag
\end{align}\\
The above convolution can be written in the form of matrix as:
$\vec{Y}=\vec{H}\vec{X}$\\
\begin{align}
    Y=\myvec{h_0&0&0&.&.&0&0\\h_1&h_0&0&.&.&0&0\\h_2&h_1&h_0&.&.&0&0\\.&.&.&.&.&.&.\\h_{n-1}&h_{n-2}&n_{n-3}&.&.&h_1&h_0\\.&.&.&.&.&.&.\\h_{m-1}&h_{m-2}&h_{m-3}&.&.&h_{m-n+1}&h_{m-n}\\0&h_{m-1}&h_{m-2}&.&.&h_{m-n+2}&h_{m-n+1}\\0&0&h_{m-1}&.&.&h_{m-n+3}&h_{m-n+2}\\.&.&.&.&.&.&.\\0&0&0&.&.&0&h_{m-1}}\myvec{x_0\\x_1\\x_2.\\.\\.\\x_{n-1}}
\end{align}
Therefore we can write equation \eqref{eq:1} in matrix form as $\vec{Y}=\vec{H}\vec{X}$ where
\begin{align}
    \vec{H}=\myvec{h_0&0&0&.&.&0&0\\h_1&h_0&0&.&.&0&0\\h_2&h_1&h_0&.&.&0&0\\.&.&.&.&.&.&.\\h_{n-1}&h_{n-2}&n_{n-3}&.&.&h_1&h_0\\.&.&.&.&.&.&.\\h_{m-1}&h_{m-2}&h_{m-3}&.&.&h_{m-n+1}&h_{m-n}\\0&h_{m-1}&h_{m-2}&.&.&h_{m-n+2}&h_{m-n+1}\\0&0&h_{m-1}&.&.&h_{m-n+3}&h_{m-n+2}\\.&.&.&.&.&.&.\\0&0&0&.&.&0&h_{m-1}}\label{eq:p3}
\end{align}
    Therefore, from question we can rewrite convolution in form of matrices as,
\begin{align}
    x_1(n) \ast h_1(n) = y(n)\\
    \implies \vec{Y}=\vec{H_1}\vec{X} \label{eq:p1}\\
    x_2(n) \ast h_2(n) = y(n)\\
    \implies \vec{Y}=\vec{H_2}\vec{X}\label{eq:p2}
\end{align}
Equating equations \eqref{eq:p1} and \eqref{eq:p2},
\begin{align}
\vec{H_1}\vec{X} = \vec{H_2}\vec{X}
\end{align}
$\vec{H_1}$ and $\vec{H_2}$ are in the form of equation \eqref{eq:p3} and we can see by comparing both matrices that:
\begin{align}
\implies \boxed{\vec{H_1} = \vec{H_2}}
\end{align}
Because the upper half of matrix is lower triangular and the values of both matrix will be equal. Similarly the matrix from below is upper triangular and both matrix will be equal by comparison.
\\\end{document}
