\documentclass[journal,12pt,twocolumn]{IEEEtran}

\usepackage{setspace}
\usepackage{gensymb}

\singlespacing


\usepackage[cmex10]{amsmath}

\usepackage{amsthm}

\usepackage{mathrsfs}
\usepackage{txfonts}
\usepackage{stfloats}
\usepackage{bm}
\usepackage{cite}
\usepackage{cases}
\usepackage{subfig}

\usepackage{longtable}
\usepackage{multirow}

\usepackage{enumitem}
\usepackage{mathtools}
\usepackage{steinmetz}
\usepackage{tikz}
\usepackage{circuitikz}
\usepackage{verbatim}
\usepackage{tfrupee}
\usepackage[breaklinks=true]{hyperref}

\usepackage{tkz-euclide}

\usetikzlibrary{calc,math}
\usepackage{listings}
    \usepackage{color}                                            %%
    \usepackage{array}                                            %%
    \usepackage{longtable}                                        %%
    \usepackage{calc}                                             %%
    \usepackage{multirow}                                         %%
    \usepackage{hhline}                                           %%
    \usepackage{ifthen}                                           %%
    \usepackage{lscape}     
\usepackage{multicol}
\usepackage{chngcntr}

\DeclareMathOperator*{\Res}{Res}

\renewcommand\thesection{\arabic{section}}
\renewcommand\thesubsection{\thesection.\arabic{subsection}}
\renewcommand\thesubsubsection{\thesubsection.\arabic{subsubsection}}

\renewcommand\thesectiondis{\arabic{section}}
\renewcommand\thesubsectiondis{\thesectiondis.\arabic{subsection}}
\renewcommand\thesubsubsectiondis{\thesubsectiondis.\arabic{subsubsection}}


\hyphenation{op-tical net-works semi-conduc-tor}
\def\inputGnumericTable{}                                 %%

\lstset{
%language=C,
frame=single, 
breaklines=true,
columns=fullflexible
}
\begin{document}


\newtheorem{theorem}{Theorem}[section]
\newtheorem{problem}{Problem}
\newtheorem{proposition}{Proposition}[section]
\newtheorem{lemma}{Lemma}[section]
\newtheorem{corollary}[theorem]{Corollary}
\newtheorem{example}{Example}[section]
\newtheorem{definition}[problem]{Definition}

\newcommand{\BEQA}{\begin{eqnarray}}
\newcommand{\EEQA}{\end{eqnarray}}
\newcommand{\define}{\stackrel{\triangle}{=}}
\bibliographystyle{IEEEtran}

\providecommand{\mbf}{\mathbf}
\providecommand{\pr}[1]{\ensuremath{\Pr\left(#1\right)}}
\providecommand{\qfunc}[1]{\ensuremath{Q\left(#1\right)}}
\providecommand{\sbrak}[1]{\ensuremath{{}\left[#1\right]}}
\providecommand{\lsbrak}[1]{\ensuremath{{}\left[#1\right.}}
\providecommand{\rsbrak}[1]{\ensuremath{{}\left.#1\right]}}
\providecommand{\brak}[1]{\ensuremath{\left(#1\right)}}
\providecommand{\lbrak}[1]{\ensuremath{\left(#1\right.}}
\providecommand{\rbrak}[1]{\ensuremath{\left.#1\right)}}
\providecommand{\cbrak}[1]{\ensuremath{\left\{#1\right\}}}
\providecommand{\lcbrak}[1]{\ensuremath{\left\{#1\right.}}
\providecommand{\rcbrak}[1]{\ensuremath{\left.#1\right\}}}
\theoremstyle{remark}
\newtheorem{rem}{Remark}
\newcommand{\sgn}{\mathop{\mathrm{sgn}}}
\providecommand{\abs}[1]{\left\vert#1\right\vert}
\providecommand{\res}[1]{\Res\displaylimits_{#1}} 
\providecommand{\norm}[1]{\left\lVert#1\right\rVert}
%\providecommand{\norm}[1]{\lVert#1\rVert}
\providecommand{\mtx}[1]{\mathbf{#1}}
\providecommand{\mean}[1]{E\left[ #1 \right]}
\providecommand{\fourier}{\overset{\mathcal{F}}{ \rightleftharpoons}}
%\providecommand{\hilbert}{\overset{\mathcal{H}}{ \rightleftharpoons}}
\providecommand{\system}{\overset{\mathcal{H}}{ \longleftrightarrow}}
	%\newcommand{\solution}[2]{\textbf{Solution:}{#1}}
\newcommand{\solution}{\noindent \textbf{Solution: }}
\newcommand{\cosec}{\,\text{cosec}\,}
\providecommand{\dec}[2]{\ensuremath{\overset{#1}{\underset{#2}{\gtrless}}}}
\newcommand{\myvec}[1]{\ensuremath{\begin{pmatrix}#1\end{pmatrix}}}
\newcommand{\mydet}[1]{\ensuremath{\begin{vmatrix}#1\end{vmatrix}}}

\numberwithin{equation}{subsection}

\makeatletter
\@addtoreset{figure}{problem}
\makeatother
\let\StandardTheFigure\thefigure
\let\vec\mathbf

\renewcommand{\thefigure}{\theproblem}

\def\putbox#1#2#3{\makebox[0in][l]{\makebox[#1][l]{}\raisebox{\baselineskip}[0in][0in]{\raisebox{#2}[0in][0in]{#3}}}}
     \def\rightbox#1{\makebox[0in][r]{#1}}
     \def\centbox#1{\makebox[0in]{#1}}
     \def\topbox#1{\raisebox{-\baselineskip}[0in][0in]{#1}}
     \def\midbox#1{\raisebox{-0.5\baselineskip}[0in][0in]{#1}}
\vspace{3cm}
\title{Challenging Problem}
\author{Sri Harsha CH}

\maketitle
\newpage

\bigskip
\renewcommand{\thefigure}{\theenumi}
\renewcommand{\thetable}{\theenumi}

\begin{abstract}
This document explains the proof of Cayley-Hamilton Theorem.
\end{abstract}

Download latex-tikz codes from 
%
\begin{lstlisting}
https://github.com/harshachinta/EE5609-Matrix-Theory/tree/master/Assignments/Assignment3
\end{lstlisting}
%
\section{Problem}
Prove Cayley-Hamilton Theorem.
\section{Explanation}
Cayley-Hamilton Theorem: Every Square matrix satisfies its own characteristic equation.\\
Proof:\\
Let $\vec{A}$ = \myvec{a_{ij}} be any square matrix of order $n$ and $\phi(\lambda) = \det({\vec{A}-\lambda \vec{I}})$ be its characteristic equation in variable $\lambda$ and $\vec{I}$ is the identity matrix of order $n$.\\
The characteristic equation of $\vec{A}$ is 
\begin{align}  
&\det({\vec{A}-\lambda \vec{I}}) = 0 \label{eq:eq_1}\\
& \mydet{a_{11}-\lambda & a_{12} & a_{13} & \cdots & a_{nn} \\ 
a_{21} & a_{22}-\lambda & a_{23}& \cdots & a_{2n} \\
\vdots & \vdots & \vdots & \ddots &\vdots \\
a_{n1} & a_{n2} & a_{n3} & \cdots & a_{nn}-\lambda \\ } = 0\\
& \implies a_0 + a_1\lambda + a_2\lambda^{2} +  \cdots+a_n\lambda^{n} = 0\label{eq:char_eq}
\end{align}
According to Cayley-Hamilton theorem, we need to prove that:
\begin{align} 
&a_0 + a_1\vec{A} + a_2\vec{A^{2}} +  \cdots+a_n\vec{A^{n}} = 0 \label{eq:to_prove}
\end{align}
we know that,
\begin{align} 
& \vec{A}(adj\vec{A}) = \det(\vec{A})\vec{I} \label{eq:adj_1} \\
&\implies (\vec{A}-\lambda\vec{I})adj(\vec{A}-\lambda\vec{I}) = \det(\vec{A}-\lambda\vec{I})\vec{I}\label{eq:adj_2}
\end{align}
From equation \eqref{eq:adj_2} we already know RHS, but we need to determine LHS and compare the coefficients,\\
From RHS of \eqref{eq:adj_2},we know that,
$adj(\vec{A}-\lambda\vec{I})$ has a polynomial of degree $n-1$.\\
\begin{align} 
& adj(\vec{A}-\lambda\vec{I}) = b_0 + b_1\lambda + b_2\lambda^{2} +  \cdots+b_{n-1}\lambda^{n-1}\label{eq:adj_3}
\end{align}
Using equation \eqref{eq:adj_3},
\begin{multline}
 (\vec{A}-\lambda\vec{I}) adj(\vec{A}-\lambda\vec{I}) \\= (\vec{A}-\lambda\vec{I})(b_0 + b_1\lambda + b_2\lambda^{2} +  \cdots+b_{n-1}\lambda^{n-1})\\
\end{multline}
\begin{multline}
= \vec{A}b_0 + \vec{A}b_1\lambda +\cdots+ \vec{A}b_{n-1}\lambda^{n-1}\\
- b_0\lambda - b_1\lambda^{2}-\cdots - b_{n-1}\lambda^{n}\\
=\vec{A}b_0 + \lambda(\vec{A}b_1 - b_0) + \lambda^{2}(\vec{A}b_2 - b_1) + \cdots - b_{n-1}\lambda^{n}  \label{eq:adj_4}
\end{multline}
Substituting equ \eqref{eq:adj_4} and \eqref{eq:char_eq}  in \eqref{eq:adj_2},
\begin{multline} \label{eq:comp}
\vec{A}b_0 + \lambda(\vec{A}b_1 - b_0) + \lambda^{2}(\vec{A}b_2 - b_1) + \cdots - b_{n-1}\lambda^{n} =\\
a_0 + a_1\lambda + a_2\lambda^{2} +  \cdots+a_n\lambda^{n}
\end{multline}
Comparing coefficients of equal powers of $\lambda$ in equation \eqref{eq:comp},
\begin{align}
& \vec{A}b_0 = a_0 \label{eq:eq1}\\
& \vec{A}b_1 - b_0 = a_1 \label{eq:eq2}\\
& \vdots \\
& \vec{A}b_{n-1} - b_{n-2} = a_{n-1} \label{eq:eq3}\\
& -b_{n-1} = a_n \label{eq:eq4}
\end{align}
Multiplying equ \eqref{eq:eq1} by $\vec{I}$, equ \eqref{eq:eq2} by $\vec{A}$, $\cdots$,  \eqref{eq:eq3} by $\vec{A^{n-1}}$ and  \eqref{eq:eq4} by $\vec{A^{n}}$, we get,
\begin{align}
& \vec{A}b_0 = a_0\vec{I} \label{eq:eq5}\\
& \vec{A^{2}}b_1 - \vec{A}b_0 = \vec{A}a_1 \label{eq:eq6}\\
& \vdots \\
& \vec{A^{n}}b_{n-1} - \vec{A^{n-1}}b_{n-2} = a_{n-1}\vec{A^{n-1}} \label{eq:eq7}\\
& -\vec{A^{n}}b_{n-1} = a_n\vec{A^{n}} \label{eq:eq8}
\end{align}
Adding all the equations above, we get:
\begin{align}
&\implies \boxed{a_0 + a_1\vec{A} + a_2\vec{A^{2}} +  \cdots+a_n\vec{A^{n}} = 0} \label{eq:final}
\end{align}
\section{Solution}

Equation \eqref{eq:final} proves Cayley-Hamilton theorem.
\\\end{document}
