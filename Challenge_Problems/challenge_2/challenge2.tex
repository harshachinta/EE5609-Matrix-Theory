\documentclass[journal,12pt,twocolumn]{IEEEtran}

\usepackage{setspace}
\usepackage{gensymb}

\singlespacing


\usepackage[cmex10]{amsmath}

\usepackage{amsthm}

\usepackage{mathrsfs}
\usepackage{txfonts}
\usepackage{stfloats}
\usepackage{bm}
\usepackage{cite}
\usepackage{cases}
\usepackage{subfig}

\usepackage{longtable}
\usepackage{multirow}

\usepackage{enumitem}
\usepackage{mathtools}
\usepackage{steinmetz}
\usepackage{tikz}
\usepackage{circuitikz}
\usepackage{verbatim}
\usepackage{tfrupee}
\usepackage[breaklinks=true]{hyperref}

\usepackage{tkz-euclide}

\usetikzlibrary{calc,math}
\usepackage{listings}
    \usepackage{color}                                            %%
    \usepackage{array}                                            %%
    \usepackage{longtable}                                        %%
    \usepackage{calc}                                             %%
    \usepackage{multirow}                                         %%
    \usepackage{hhline}                                           %%
    \usepackage{ifthen}                                           %%
    \usepackage{lscape}     
\usepackage{multicol}
\usepackage{chngcntr}

\DeclareMathOperator*{\Res}{Res}

\renewcommand\thesection{\arabic{section}}
\renewcommand\thesubsection{\thesection.\arabic{subsection}}
\renewcommand\thesubsubsection{\thesubsection.\arabic{subsubsection}}

\renewcommand\thesectiondis{\arabic{section}}
\renewcommand\thesubsectiondis{\thesectiondis.\arabic{subsection}}
\renewcommand\thesubsubsectiondis{\thesubsectiondis.\arabic{subsubsection}}


\hyphenation{op-tical net-works semi-conduc-tor}
\def\inputGnumericTable{}                                 %%

\lstset{
%language=C,
frame=single, 
breaklines=true,
columns=fullflexible
}
\begin{document}


\newtheorem{theorem}{Theorem}[section]
\newtheorem{problem}{Problem}
\newtheorem{proposition}{Proposition}[section]
\newtheorem{lemma}{Lemma}[section]
\newtheorem{corollary}[theorem]{Corollary}
\newtheorem{example}{Example}[section]
\newtheorem{definition}[problem]{Definition}

\newcommand{\BEQA}{\begin{eqnarray}}
\newcommand{\EEQA}{\end{eqnarray}}
\newcommand{\define}{\stackrel{\triangle}{=}}
\bibliographystyle{IEEEtran}

\providecommand{\mbf}{\mathbf}
\providecommand{\pr}[1]{\ensuremath{\Pr\left(#1\right)}}
\providecommand{\qfunc}[1]{\ensuremath{Q\left(#1\right)}}
\providecommand{\sbrak}[1]{\ensuremath{{}\left[#1\right]}}
\providecommand{\lsbrak}[1]{\ensuremath{{}\left[#1\right.}}
\providecommand{\rsbrak}[1]{\ensuremath{{}\left.#1\right]}}
\providecommand{\brak}[1]{\ensuremath{\left(#1\right)}}
\providecommand{\lbrak}[1]{\ensuremath{\left(#1\right.}}
\providecommand{\rbrak}[1]{\ensuremath{\left.#1\right)}}
\providecommand{\cbrak}[1]{\ensuremath{\left\{#1\right\}}}
\providecommand{\lcbrak}[1]{\ensuremath{\left\{#1\right.}}
\providecommand{\rcbrak}[1]{\ensuremath{\left.#1\right\}}}
\theoremstyle{remark}
\newtheorem{rem}{Remark}
\newcommand{\sgn}{\mathop{\mathrm{sgn}}}
\providecommand{\abs}[1]{\left\vert#1\right\vert}
\providecommand{\res}[1]{\Res\displaylimits_{#1}} 
\providecommand{\norm}[1]{\left\lVert#1\right\rVert}
%\providecommand{\norm}[1]{\lVert#1\rVert}
\providecommand{\mtx}[1]{\mathbf{#1}}
\providecommand{\mean}[1]{E\left[ #1 \right]}
\providecommand{\fourier}{\overset{\mathcal{F}}{ \rightleftharpoons}}
%\providecommand{\hilbert}{\overset{\mathcal{H}}{ \rightleftharpoons}}
\providecommand{\system}{\overset{\mathcal{H}}{ \longleftrightarrow}}
	%\newcommand{\solution}[2]{\textbf{Solution:}{#1}}
\newcommand{\solution}{\noindent \textbf{Solution: }}
\newcommand{\cosec}{\,\text{cosec}\,}
\providecommand{\dec}[2]{\ensuremath{\overset{#1}{\underset{#2}{\gtrless}}}}
\newcommand{\myvec}[1]{\ensuremath{\begin{pmatrix}#1\end{pmatrix}}}
\newcommand{\mydet}[1]{\ensuremath{\begin{vmatrix}#1\end{vmatrix}}}

\numberwithin{equation}{subsection}

\makeatletter
\@addtoreset{figure}{problem}
\makeatother
\let\StandardTheFigure\thefigure
\let\vec\mathbf

\renewcommand{\thefigure}{\theproblem}

\def\putbox#1#2#3{\makebox[0in][l]{\makebox[#1][l]{}\raisebox{\baselineskip}[0in][0in]{\raisebox{#2}[0in][0in]{#3}}}}
     \def\rightbox#1{\makebox[0in][r]{#1}}
     \def\centbox#1{\makebox[0in]{#1}}
     \def\topbox#1{\raisebox{-\baselineskip}[0in][0in]{#1}}
     \def\midbox#1{\raisebox{-0.5\baselineskip}[0in][0in]{#1}}
\vspace{3cm}
\title{Challenge Problem}
\author{Sri Harsha CH}

\maketitle
\newpage

\bigskip
\renewcommand{\thefigure}{\theenumi}
\renewcommand{\thetable}{\theenumi}

\begin{abstract}
This document explains the conditions when pseudo inverse does not exist.
\end{abstract}

Download latex-tikz codes from 
%
\begin{lstlisting}
https://github.com/harshachinta/EE5609-Matrix-Theory/tree/master/Challenges/Challenge2
\end{lstlisting}
%
\section{Problem}
What are the conditions in which pseudo inverse exists?.
\section{Explanation}
This problem is a part of
\begin{lstlisting}
https://github.com/Zeeshan-IITH/IITH-EE5609/blob/master/assignment2/assignment2.pdf
\end{lstlisting}
 where one other method of finding the solution ($\lambda_1$ and $\lambda_2$) would be by computing the inverse from equation (3.0.8) mentioned in the above document.\\

The matrix $\vec{M}$ was defined as:
\begin{align}
    \vec{M}=\myvec{\vec{m_1}^T\\\vec{m_2}^T}
\end{align}
where $\vec{m_1}$,$\vec{m_2}$ are vectors perpendicular to line ${AB}$.
and,
\begin{align}
    \vec{M}\vec{M}^T\begin{pmatrix}\lambda_1\\-\lambda_2\end{pmatrix}=\vec{M}\vec{(x_2-x_1)}
\end{align}
where $\lambda_1$ and $\lambda_2$ are points on line.\\
Here $\lambda_1$ and $\lambda_2$ can be computed by taking the inverse of $\vec{M}\vec{M}^T$.\\
\begin{align}
    \begin{pmatrix}\lambda_1\\-\lambda_2\end{pmatrix}=(\vec{M}\vec{M}^T)^{-1}\vec{M}\vec{(x_2-x_1)}
\end{align}
The inverse $(\vec{M}\vec{M}^T)^{-1}$ can be written as:
\begin{align}
    (\vec{M}\vec{M}^T)^{-1} = (\vec{M}^T)^{-1}\vec{M}^{-1} \label{eq:eq_1}
\end{align}
As $\vec{M}$ consists of vectors $\vec{m_1}$,$\vec{m_2}$ which are of order $n\times1$, the order of matrix $\vec{M}$ would be $2 \times n$.\\ That is matrix $\vec{M}$ is a rectangular matrix and computing inverse of a rectangular matrix is not possible.\\ Hence we use pseudo inverse method to compute inverse.
\begin{align}
    \vec{M}=\myvec{\vec{m_1}^T\\\vec{m_2}^T}
\end{align}
As, $\vec{M}$ is of order $2 \times n$, the rank of this matrix will be $\min(2, n)$. $\implies$ rank will be less than or equal to 2, that is it depends on linear independence of rows of matrix.\\
\\
\textbf{Condition1}: If $\vec{m_1}$ and $\vec{m_2}$ are not parallel, then the rows of $\vec{M}$ are linearly independent, that is rank of $\vec{M}$ is 2, which is full row rank.\\
Example:
\begin{align}
    \vec{m_1} = \myvec{1\\-1\\1}
    \vec{m_2} = \myvec{2\\1\\2}\\
    \vec{M} = \myvec{1&-1&1\\2&1&2}
\end{align}
As $\vec{M}$ is full row rank, right pseudo inverse exists and can be computed as,\\
\begin{align}
    \vec{M}^T(\vec{M}\vec{M}^T)^{-1}
\end{align}
\\
\textbf{Condition2}: If $\vec{m_1}$ and $\vec{m_2}$ are parallel, then the rows of $\vec{M}$ are linearly dependent, that is rank of $\vec{M}$ is 1 or 0, which is rank deficient matrix. \textbf{Therefore, pseudo inverse cannot be computed in such situation}.\\
Example:
\begin{align}
    \vec{m_1} = \myvec{1\\1\\1}
    \vec{m_2} = \myvec{2\\2\\2}\\
    \vec{M} = \myvec{1&1&1\\2&2&2}
\end{align}
Similarly, if a rectangular matrix has full column rank then left pseudo inverse exists and this is usually for tall matrices.\\

There is one more way to compute inverse without splitting $(\vec{M}\vec{M}^T)^{-1}$ as in equation \eqref{eq:eq_1}. In this case $(\vec{M}\vec{M}^T)$ is a square matrix of order $2\times 2$. As it is a square matrix, if rows or columns are independent then the matrix would not be singular and inverse and pseudo inverse of such a matrix would be same. But if it singular then inverse does not exist.

\section{Solution}

If a matrix is full column rank then left pseudo inverse exist. If it is a full row rank then right pseudo inverse exist. If a matrix is square and invertible then inverse and pseudo inverse would be same.

\\\end{document}