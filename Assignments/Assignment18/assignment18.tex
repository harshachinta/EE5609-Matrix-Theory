\documentclass[journal,12pt,twocolumn]{IEEEtran}

\usepackage{setspace}
\usepackage{gensymb}

\singlespacing


\usepackage[cmex10]{amsmath}

\usepackage{amsthm}

\usepackage{mathrsfs}
\usepackage{txfonts}
\usepackage{stfloats}
\usepackage{bm}
\usepackage{cite}
\usepackage{cases}
\usepackage{subfig}

\usepackage{longtable}
\usepackage{multirow}

\usepackage{enumitem}
\usepackage{mathtools}
\usepackage{steinmetz}
\usepackage{tikz}
\usepackage{circuitikz}
\usepackage{verbatim}
\usepackage{tfrupee}
\usepackage[breaklinks=true]{hyperref}

\usepackage{tkz-euclide}

\usetikzlibrary{calc,math}
\usepackage{listings}
    \usepackage{color}                                            %%
    \usepackage{array}                                            %%
    \usepackage{longtable}                                        %%
    \usepackage{calc}                                             %%
    \usepackage{multirow}                                         %%
    \usepackage{hhline}                                           %%
    \usepackage{ifthen}                                           %%
    \usepackage{lscape}     
\usepackage{multicol}
\usepackage{chngcntr}

\DeclareMathOperator*{\Res}{Res}
\DeclareMathOperator{\range}{range}

\renewcommand\thesection{\arabic{section}}
\renewcommand\thesubsection{\thesection.\arabic{subsection}}
\renewcommand\thesubsubsection{\thesubsection.\arabic{subsubsection}}

\renewcommand\thesectiondis{\arabic{section}}
\renewcommand\thesubsectiondis{\thesectiondis.\arabic{subsection}}
\renewcommand\thesubsubsectiondis{\thesubsectiondis.\arabic{subsubsection}}


\hyphenation{op-tical net-works semi-conduc-tor}
\def\inputGnumericTable{}                                 %%

\lstset{
%language=C,
frame=single, 
breaklines=true,
columns=fullflexible
}
\begin{document}


\newtheorem{theorem}{Theorem}[section]
\newtheorem{problem}{Problem}
\newtheorem{proposition}{Proposition}[section]
\newtheorem{lemma}{Lemma}[section]
\newtheorem{corollary}[theorem]{Corollary}
\newtheorem{example}{Example}[section]
\newtheorem{definition}[problem]{Definition}

\newcommand{\BEQA}{\begin{eqnarray}}
\newcommand{\EEQA}{\end{eqnarray}}
\newcommand{\define}{\stackrel{\triangle}{=}}
\bibliographystyle{IEEEtran}

\providecommand{\mbf}{\mathbf}
\providecommand{\pr}[1]{\ensuremath{\Pr\left(#1\right)}}
\providecommand{\qfunc}[1]{\ensuremath{Q\left(#1\right)}}
\providecommand{\sbrak}[1]{\ensuremath{{}\left[#1\right]}}
\providecommand{\lsbrak}[1]{\ensuremath{{}\left[#1\right.}}
\providecommand{\rsbrak}[1]{\ensuremath{{}\left.#1\right]}}
\providecommand{\brak}[1]{\ensuremath{\left(#1\right)}}
\providecommand{\lbrak}[1]{\ensuremath{\left(#1\right.}}
\providecommand{\rbrak}[1]{\ensuremath{\left.#1\right)}}
\providecommand{\cbrak}[1]{\ensuremath{\left\{#1\right\}}}
\providecommand{\lcbrak}[1]{\ensuremath{\left\{#1\right.}}
\providecommand{\rcbrak}[1]{\ensuremath{\left.#1\right\}}}
\theoremstyle{remark}
\newtheorem{rem}{Remark}
\newcommand{\sgn}{\mathop{\mathrm{sgn}}}
\providecommand{\abs}[1]{\left\vert#1\right\vert}
\providecommand{\res}[1]{\Res\displaylimits_{#1}} 
\providecommand{\norm}[1]{\left\lVert#1\right\rVert}
%\providecommand{\norm}[1]{\lVert#1\rVert}
\providecommand{\mtx}[1]{\mathbf{#1}}
\providecommand{\mean}[1]{E\left[ #1 \right]}
\providecommand{\fourier}{\overset{\mathcal{F}}{ \rightleftharpoons}}
%\providecommand{\hilbert}{\overset{\mathcal{H}}{ \rightleftharpoons}}
\providecommand{\system}{\overset{\mathcal{H}}{ \longleftrightarrow}}
	%\newcommand{\solution}[2]{\textbf{Solution:}{#1}}
\newcommand{\solution}{\noindent \textbf{Solution: }}
\newcommand{\cosec}{\,\text{cosec}\,}
\providecommand{\dec}[2]{\ensuremath{\overset{#1}{\underset{#2}{\gtrless}}}}
\newcommand{\myvec}[1]{\ensuremath{\begin{pmatrix}#1\end{pmatrix}}}
\newcommand{\mydet}[1]{\ensuremath{\begin{vmatrix}#1\end{vmatrix}}}

\numberwithin{equation}{subsection}

\makeatletter
\@addtoreset{figure}{problem}
\makeatother
\let\StandardTheFigure\thefigure
\let\vec\mathbf

\renewcommand{\thefigure}{\theproblem}

\def\putbox#1#2#3{\makebox[0in][l]{\makebox[#1][l]{}\raisebox{\baselineskip}[0in][0in]{\raisebox{#2}[0in][0in]{#3}}}}
     \def\rightbox#1{\makebox[0in][r]{#1}}
     \def\centbox#1{\makebox[0in]{#1}}
     \def\topbox#1{\raisebox{-\baselineskip}[0in][0in]{#1}}
     \def\midbox#1{\raisebox{-0.5\baselineskip}[0in][0in]{#1}}
\vspace{3cm}
\title{Assignment 18}
\author{Sri Harsha CH}

\maketitle
\newpage

\bigskip
\renewcommand{\thefigure}{\theenumi}
\renewcommand{\thetable}{\theenumi}

\begin{abstract}
This document explains the representation of transformations by matrix.
\end{abstract}

Download all python codes from 
\begin{lstlisting}
https://github.com/harshachinta/EE5609-Matrix-Theory/tree/master/Assignments/Assignment18/code
\end{lstlisting}
%
and latex-tikz codes from 
%
\begin{lstlisting}
https://github.com/harshachinta/EE5609-Matrix-Theory/tree/master/Assignments/Assignment18
\end{lstlisting}
%
\section{Problem}
Let $\vec{F}$ be a subfield of the field of complex numbers and let $\vec{V}$ be any vector space over $\vec{F}$. Suppose that f and g are linear functionals on $\vec{V}$ such that the function $h$ defined by $h(\alpha) =f(\alpha) g(\alpha)$ is also a linear functional on $\vec{V}$. Prove that either $f=0$ or $g=0$.

\section{Explanation}
Refer Table \ref{table:1}.


\begin{table*}[ht!]
\begin{center}
\begin{tabular}{|l|l|}
\hline
\textbf{Given} & \textbf{Derivation} \\[0.5ex]
\hline
\text{$f$, $g$, $h$ are linear functionals of $\vec{V}$} & 
\text{By contradiction, let us assume $f \not= 0$ and $g \not= 0$. For all $\vec{v} \in \vec{V}$}\\
& \parbox{10cm}{\begin{align}
    h(\vec{v}) &= f(\vec{v}) g(\vec{v}) \\
    h(2\vec{v}) &= f(2\vec{v}) g(2\vec{v})\\
    &=2f(\vec{v}) 2g(\vec{v})\\
    &=4f(\vec{v})g(\vec{v}) \label{eq:eq1}
\end{align}} \\
& Similarly,\\
& \parbox{10cm}{\begin{align}
    h(2\vec{v}) &= 2h(\vec{v}) \\
    &= 2f(\vec{v})g(\vec{v}) \label{eq:eq2}
\end{align}} \\
& From equation \eqref{eq:eq1} and \eqref{eq:eq2},\\
& \parbox{10cm}{\begin{align}
    \implies 4f(\vec{v})g(\vec{v}) = 2f(\vec{v})g(\vec{v})\\
    \implies f(\vec{v}).g(\vec{v})=0 \label{eq:eq3}
\end{align}}
\\ [0.5ex]
\hline
\text{Choosing Basis} & 
\text{Let $\vec{B}$ be a basis for $\vec{V}$. Let,}\\
& \parbox{10cm}{\begin{align}
    \vec{B_1} &= \{\vec{b} \in \vec{B}\ \vert \ f(\vec{b})=0\}, \\
    \vec{B_2} &= \{\vec{b} \in \vec{B}\ \vert \ g(\vec{b})=0\}
\end{align}} \\
& Since,\\
& \parbox{10cm}{\begin{align}
    f(\vec{b}).g(\vec{b})=0 \quad \forall \vec{b} \in \vec{B}\\
    \implies f(\vec{b})=0 \text{ or } g(\vec{b})=0\\
    \implies \vec{b} \in \vec{B_1} \text{ or } \vec{b} \in \vec{B_2}
\end{align}}
\\ [0.5ex]
\hline
\text{Choosing $\vec{b_1}$ and $\vec{b_2}$ from basis} & 
Let us choose $\vec{b_1} \in \vec{B_1}-\vec{B_2}$ and $\vec{b_2} \in \vec{B_2}-\vec{B_1}$ \\
& $\implies$ $f(\vec{b_2}) \not = 0$ and $g(\vec{b_1}) \not = 0$\\
& \parbox{10cm}{\begin{align}
    f(\vec{b_1}+\vec{b_2}).g(\vec{b_1}+\vec{b_2})= (f(\vec{b_1})+ f(\vec{b_2})).(g(\vec{b_1})+ g(\vec{b_2}))\\
    =f(\vec{b_1}).g(\vec{b_1}) + f(\vec{b_1}).g(\vec{b_2}) +f(\vec{b_2}).g(\vec{b_1}) + f(\vec{b_2}).g(\vec{b_2})\\
    = 0 + 0 + f(\vec{b_2}).g(\vec{b_1}) + 0\\
    = f(\vec{b_2}).g(\vec{b_1}) \not= 0 \label{eq:eq6}
\end{align}}\\
& Equation \eqref{eq:eq6} is contradiction to the fact that $f(\vec{v}).g(\vec{v})=0$.\\
& $\implies \boxed{f=0 \text{ or } g=0}$
\\ [0.5ex]
\hline
\end{tabular}
\caption{Expanation}
\label{table:1}
\end{center}
\vspace{-0.5cm}
\end{table*}
\end{document}
