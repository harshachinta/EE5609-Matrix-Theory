\documentclass[journal,12pt]{IEEEtran}
\usepackage{longtable}
\usepackage{setspace}
\usepackage{gensymb}
\singlespacing
\usepackage[cmex10]{amsmath}
\newcommand\myemptypage{
	\null
	\thispagestyle{empty}
	\addtocounter{page}{-1}
	\newpage
}
\usepackage{amsthm}
\usepackage{mdframed}
\usepackage{mathrsfs}
\usepackage{txfonts}
\usepackage{stfloats}
\usepackage{bm}
\usepackage{cite}
\usepackage{cases}
\usepackage{subfig}

\usepackage{longtable}
\usepackage{multirow}

\usepackage{enumitem}
\usepackage{mathtools}
\usepackage{steinmetz}
\usepackage{tikz}
\usepackage{circuitikz}
\usepackage{verbatim}
\usepackage{tfrupee}
\usepackage[breaklinks=true]{hyperref}
\usepackage{graphicx}
\usepackage{tkz-euclide}

\usetikzlibrary{calc,math}
\usepackage{listings}
    \usepackage{color}                                            %%
    \usepackage{array}                                            %%
    \usepackage{longtable}                                        %%
    \usepackage{calc}                                             %%
    \usepackage{multirow}                                         %%
    \usepackage{hhline}                                           %%
    \usepackage{ifthen}                                           %%
    \usepackage{lscape}     
\usepackage{multicol}
\usepackage{chngcntr}

\DeclareMathOperator*{\Res}{Res}

\renewcommand\thesection{\arabic{section}}
\renewcommand\thesubsection{\thesection.\arabic{subsection}}
\renewcommand\thesubsubsection{\thesubsection.\arabic{subsubsection}}

\renewcommand\thesectiondis{\arabic{section}}
\renewcommand\thesubsectiondis{\thesectiondis.\arabic{subsection}}
\renewcommand\thesubsubsectiondis{\thesubsectiondis.\arabic{subsubsection}}


\hyphenation{op-tical net-works semi-conduc-tor}
\def\inputGnumericTable{}                                 %%

\lstset{
%language=C,
frame=single, 
breaklines=true,
columns=fullflexible
}
\begin{document}
\onecolumn

\newtheorem{theorem}{Theorem}[section]
\newtheorem{problem}{Problem}
\newtheorem{proposition}{Proposition}[section]
\newtheorem{lemma}{Lemma}[section]
\newtheorem{corollary}[theorem]{Corollary}
\newtheorem{example}{Example}[section]
\newtheorem{definition}[problem]{Definition}

\newcommand{\BEQA}{\begin{eqnarray}}
\newcommand{\EEQA}{\end{eqnarray}}
\newcommand{\define}{\stackrel{\triangle}{=}}
\bibliographystyle{IEEEtran}
\raggedbottom
\setlength{\parindent}{0pt}
\providecommand{\mbf}{\mathbf}
\providecommand{\pr}[1]{\ensuremath{\Pr\left(#1\right)}}
\providecommand{\qfunc}[1]{\ensuremath{Q\left(#1\right)}}
\providecommand{\sbrak}[1]{\ensuremath{{}\left[#1\right]}}
\providecommand{\lsbrak}[1]{\ensuremath{{}\left[#1\right.}}
\providecommand{\rsbrak}[1]{\ensuremath{{}\left.#1\right]}}
\providecommand{\brak}[1]{\ensuremath{\left(#1\right)}}
\providecommand{\lbrak}[1]{\ensuremath{\left(#1\right.}}
\providecommand{\rbrak}[1]{\ensuremath{\left.#1\right)}}
\providecommand{\cbrak}[1]{\ensuremath{\left\{#1\right\}}}
\providecommand{\lcbrak}[1]{\ensuremath{\left\{#1\right.}}
\providecommand{\rcbrak}[1]{\ensuremath{\left.#1\right\}}}
\theoremstyle{remark}
\newtheorem{rem}{Remark}
\newcommand{\sgn}{\mathop{\mathrm{sgn}}}
\providecommand{\abs}[1]{\left\vert#1\right\vert}
\providecommand{\res}[1]{\Res\displaylimits_{#1}} 
\providecommand{\norm}[1]{\left\lVert#1\right\rVert}
%\providecommand{\norm}[1]{\lVert#1\rVert}
\providecommand{\mtx}[1]{\mathbf{#1}}
\providecommand{\mean}[1]{E\left[ #1 \right]}
\providecommand{\fourier}{\overset{\mathcal{F}}{ \rightleftharpoons}}
%\providecommand{\hilbert}{\overset{\mathcal{H}}{ \rightleftharpoons}}
\providecommand{\system}{\overset{\mathcal{H}}{ \longleftrightarrow}}
	%\newcommand{\solution}[2]{\textbf{Solution:}{#1}}
\newcommand{\solution}{\noindent \textbf{Solution: }}
\newcommand{\cosec}{\,\text{cosec}\,}
\providecommand{\dec}[2]{\ensuremath{\overset{#1}{\underset{#2}{\gtrless}}}}
\newcommand{\myvec}[1]{\ensuremath{\begin{pmatrix}#1\end{pmatrix}}}
\newcommand{\mydet}[1]{\ensuremath{\begin{vmatrix}#1\end{vmatrix}}}
\numberwithin{equation}{subsection}
\makeatletter
\@addtoreset{figure}{problem}
\makeatother
\let\StandardTheFigure\thefigure
\let\vec\mathbf
\renewcommand{\thefigure}{\theproblem}
\def\putbox#1#2#3{\makebox[0in][l]{\makebox[#1][l]{}\raisebox{\baselineskip}[0in][0in]{\raisebox{#2}[0in][0in]{#3}}}}
     \def\rightbox#1{\makebox[0in][r]{#1}}
     \def\centbox#1{\makebox[0in]{#1}}
     \def\topbox#1{\raisebox{-\baselineskip}[0in][0in]{#1}}
     \def\midbox#1{\raisebox{-0.5\baselineskip}[0in][0in]{#1}}
\vspace{3cm}
\title{Assignment 20}
\author{Sri Harsha CH}
\maketitle
\bigskip
\renewcommand{\thefigure}{\theenumi}
\renewcommand{\thetable}{\theenumi}
Download latex-tikz codes from 
%
\begin{lstlisting}
https://github.com/harshachinta/EE5609-Matrix-Theory/tree/master/Assignments/Assignment20
\end{lstlisting}
%
 
\section{Problem}
Let $\vec{N_1}$ and $\vec{N_2}$ be $6 \times 6$ nilpotent matrices over the field $\vec{F}$. Suppose that $\vec{N_1}$ and $\vec{N_2}$ have the same minimal polynomial and the same nullity. Prove that $\vec{N_1}$ and $\vec{N_2}$ are similar. Show that this is not true for $7 \times 7$ nilpotent matrices.
%
\section{\textbf{Explanation}}

\begin{longtable}{|p{5cm}|p{13cm}|}
\hline
\textbf{Statement} &\textbf{Derivation}\\
\hline 
\text{Given} & 
$\vec{N_1}$ and $\vec{N_2}$ be $6 \times 6$ nilpotent matrices. \\
& $\vec{N_1}$ and $\vec{N_2}$ have the same minimal polynomial and the same nullity.\\
& To prove $\vec{N_1}$ and $\vec{N_2}$ are similar.
\\ [0.5ex]
\hline
\text{From given statement} & 
\text{Two matrices are similar if they have the same Jordan Canonical form.}\\
& 1. As $\vec{N_1}$ and $\vec{N_2}$ are nilpotent matrices, $0$ is the only eigen value.\\
& 2. As minimal polynomial is same, $p\vec{N_1} = p\vec{N_2}$, the two matrices should have the same maximum block size.\\
& 3. As they have same nullity, they will have same total number of blocks.\\
&\\
&If $\vec{J_1}$ and $\vec{J_2}$ are similar, then $\vec{N_1}$ and $\vec{N_2}$ are similar.\\
&\\
& Let us consider all the possibilities for the dimensions of the block matrices for both Jordan forms.\\ [0.5ex]
\hline
Matrix size - 6 and Jordan size - 6 & If Jordan form $\vec{J_1}$ consists of one block of dimension 6, then by $(3)$ above $\vec{J_2}$ also has one block of dimension 6.\\&
\parbox{12cm}{\begin{align*}
    \vec{J_1}=\myvec{\vec{J}_{11}} \quad \vec{J_2}=\myvec{\vec{J}_{21}}\\
    \vec{J}_{11}: 6\times6 \quad \vec{J}_{21}: 6\times6\\
    \vec{J}_{11}, \vec{J}_{21} \text{ are similar }, \\
    \vec{J}_{1}, \vec{J}_{2} \text{ are similar }\\
    \implies \vec{N}_{1}, \vec{N}_{2} \text{ are similar }
\end{align*}}\\
\hline
 Matrix size - 6 and Jordan size - 5 + 1 & If Jordan form $\vec{J_1}$ consists of one block of dimension 5 and other 1, then by $(2)$, $\vec{J_2}$ also has same maximum block of dimension 5 and by $(3)$ have other block of size 1.
\\&
\parbox{12cm}{\begin{align*}
    \vec{J_1}=\myvec{\vec{J}_{11} & 0 \\ 0&\vec{J}_{12}} \quad \vec{J_2}=\myvec{\vec{J}_{21} & 0 \\ 0&\vec{J}_{22}}\\
    \vec{J}_{11}: 5\times5 \quad \vec{J}_{21}: 5\times5\\
    \vec{J}_{12}: 1\times1 \quad \vec{J}_{22}: 1\times1\\
    \vec{J}_{11}, \vec{J}_{21} \text{ and } \vec{J}_{12}, \vec{J}_{22} \text{ are similar }, \\
    \vec{J}_{1}, \vec{J}_{2} \text{ are similar }\\
    \implies \vec{N}_{1}, \vec{N}_{2} \text{ are similar }
\end{align*}}\\
\hline
Matrix size - 6, Jordan size - 4+2, Jordan size - 4+1+1 & Although there are two different possibilities for Jordan blocks,\\& From (2) $\vec{J}_{11}, \vec{J}_{21} $ are of dimension 4, \\& From (3) $\vec{J}$ have same number of Jordan blocks
\\& Case 1: 
\\&
\parbox{12cm}{\begin{align*}
    \vec{J_1}=\myvec{\vec{J}_{11} & 0 \\ 0&\vec{J}_{12}} \quad \vec{J_2}=\myvec{\vec{J}_{21} & 0 \\ 0&\vec{J}_{22}}\\
    \vec{J}_{11}: 4\times4 \quad \vec{J}_{21}: 4\times4\\
    \vec{J}_{12}: 2\times2 \quad \vec{J}_{22}: 2\times2\\
    \vec{J}_{11}, \vec{J}_{21} \text{ and } \vec{J}_{12}, \vec{J}_{22} \text{ are similar }, \\
    \vec{J}_{1}, \vec{J}_{2} \text{ are similar }\\
    \implies \vec{N}_{1}, \vec{N}_{2} \text{ are similar }
\end{align*}}
\\& Case 2: 
\\&
\parbox{12cm}{\begin{align*}
    \vec{J_1}=\myvec{\vec{J}_{11} & 0&0 \\ 0&\vec{J}_{12}&0\\0&0&\vec{J}_{13}} \quad \vec{J_2}=\myvec{\vec{J}_{21} & 0&0 \\ 0&\vec{J}_{22}&0\\0&0&\vec{J}_{23}}\\
    \vec{J}_{11}: 4\times4 \quad \vec{J}_{21}: 4\times4\\
    \vec{J}_{12}: 1\times1 \quad \vec{J}_{22}: 1\times1\\
    \vec{J}_{13}: 1\times1 \quad \vec{J}_{23}: 1\times1\\
    \vec{J}_{11}, \vec{J}_{21}\quad \vec{J}_{12}, \vec{J}_{22}  \text{ and } \vec{J}_{13}, \vec{J}_{23} \text{ are similar }, \\
    \vec{J}_{1}, \vec{J}_{2} \text{ are similar }\\
    \implies \vec{N}_{1}, \vec{N}_{2} \text{ are similar }
\end{align*}}\\
\hline
Matrix size - 6, Jordan size - 3+3, Jordan size - 3+2+1, Jordan size - 3+1+1+1 & There are three different possibilities for Jordan blocks,
\\& From (2) $\vec{J}_{11}, \vec{J}_{21} $ are of dimension 3, \\& From (3) $\vec{J}$ have same number of Jordan blocks
\\& Case 1: 
\\&
\parbox{12cm}{\begin{align*}
    \vec{J_1}=\myvec{\vec{J}_{11} & 0 \\ 0&\vec{J}_{12}} \quad \vec{J_2}=\myvec{\vec{J}_{21} & 0 \\ 0&\vec{J}_{22}}\\
    \vec{J}_{11}: 3\times3 \quad \vec{J}_{21}: 3\times3\\
    \vec{J}_{12}: 3\times3 \quad \vec{J}_{22}: 3\times3\\
    \vec{J}_{11}, \vec{J}_{21} \text{ and } \vec{J}_{12}, \vec{J}_{22} \text{ are similar }, \\
    \vec{J}_{1}, \vec{J}_{2} \text{ are similar }\\
    \implies \vec{N}_{1}, \vec{N}_{2} \text{ are similar }
\end{align*}}
\\& Case 2: 
\\&
\parbox{12cm}{\begin{align*}
    \vec{J_1}=\myvec{\vec{J}_{11} & 0&0 \\ 0&\vec{J}_{12}&0\\0&0&\vec{J}_{13}} \quad \vec{J_2}=\myvec{\vec{J}_{21} & 0&0 \\ 0&\vec{J}_{22}&0\\0&0&\vec{J}_{23}}\\
    \vec{J}_{11}: 3\times3 \quad \vec{J}_{21}: 3\times3\\
    \vec{J}_{12}: 2\times2 \quad \vec{J}_{22}: 2\times2\\
    \vec{J}_{13}: 1\times1 \quad \vec{J}_{23}: 1\times1\\
    \vec{J}_{11}, \vec{J}_{21}\quad \vec{J}_{12}, \vec{J}_{22}  \text{ and } \vec{J}_{13}, \vec{J}_{23} \text{ are similar }, \\
    \vec{J}_{1}, \vec{J}_{2} \text{ are similar }\\
    \implies \vec{N}_{1}, \vec{N}_{2} \text{ are similar }
\end{align*}}
\\& Case 3: 
\\&
\parbox{12cm}{\begin{align*}
    \vec{J_1}=\myvec{\vec{J}_{11} & 0&0&0 \\ 0&\vec{J}_{12}&0&0\\0&0&\vec{J}_{13}&0\\0&0&0&\vec{J}_{14}} \quad \vec{J_2}=\myvec{\vec{J}_{21} & 0&0&0 \\ 0&\vec{J}_{22}&0&0\\0&0&\vec{J}_{23}&0\\0&0&0&\vec{J}_{24}}\\
    \vec{J}_{11}: 3\times3 \quad \vec{J}_{21}: 3\times3\\
    \vec{J}_{12}: 1\times1 \quad \vec{J}_{22}: 1\times1\\
    \vec{J}_{13}: 1\times1 \quad \vec{J}_{23}: 1\times1\\
    \vec{J}_{14}: 1\times1 \quad \vec{J}_{24}: 1\times1\\
    \vec{J}_{11}, \vec{J}_{21}\quad \vec{J}_{12}, \vec{J}_{22}  \text{ and } \vec{J}_{13}, \vec{J}_{23} \quad \vec{J}_{14}, \vec{J}_{24} \text{ are similar }, \\
    \vec{J}_{1}, \vec{J}_{2} \text{ are similar }\\
    \implies \vec{N}_{1}, \vec{N}_{2} \text{ are similar }
\end{align*}}\\
\hline
Matrix size - 6, Jordan size - 2+2+2, Jordan size - 2+2+1+1, Jordan size - 2+1+1+1+1  & There are three different possibilities for Jordan blocks,
\\& From (2) $\vec{J}_{11}, \vec{J}_{21} $ are of dimension 2, \\& From (3) $\vec{J}$ have same number of Jordan blocks
\\& Case 1: 
\\&
\parbox{12cm}{\begin{align*}
    \vec{J_1}=\myvec{\vec{J}_{11} & 0&0\\ 0&\vec{J}_{12}&0 \\ 0&0&\vec{J_{13}}} \quad \vec{J_2}=\myvec{\vec{J}_{21} & 0&0\\ 0&\vec{J}_{22}&0 \\ 0&0&\vec{J_{23}}}\\
    \vec{J}_{11}: 2\times2 \quad \vec{J}_{21}: 2\times2\\
    \vec{J}_{12}: 2\times2 \quad \vec{J}_{22}: 2\times2\\
    \vec{J}_{13}: 2\times2 \quad \vec{J}_{23}: 2\times2\\
    \vec{J}_{11}, \vec{J}_{21} \quad \vec{J}_{12}, \vec{J}_{22} \text{ and } \vec{J}_{13}, \vec{J}_{23} \text{ are similar }, \\
    \vec{J}_{1}, \vec{J}_{2} \text{ are similar }\\
    \implies \vec{N}_{1}, \vec{N}_{2} \text{ are similar }
\end{align*}}
\\& Case 2: 
\\&
\parbox{12cm}{\begin{align*}
    \vec{J_1}=\myvec{\vec{J}_{11} & 0&0&0 \\ 0&\vec{J}_{12}&0&0\\0&0&\vec{J}_{13}&0\\0&0&0&\vec{J}_{14}} \quad \vec{J_2}=\myvec{\vec{J}_{21} & 0&0&0 \\ 0&\vec{J}_{22}&0&0\\0&0&\vec{J}_{23}&0\\0&0&0&\vec{J}_{24}}\\
    \vec{J}_{11}: 2\times2 \quad \vec{J}_{21}: 2\times2\\
    \vec{J}_{12}: 2\times2 \quad \vec{J}_{22}: 2\times2\\
    \vec{J}_{13}: 1\times1 \quad \vec{J}_{23}: 1\times1\\
    \vec{J}_{14}: 1\times1 \quad \vec{J}_{24}: 1\times1\\
    \vec{J}_{11}, \vec{J}_{21}\quad \vec{J}_{12}, \vec{J}_{22}  \text{ and } \vec{J}_{13}, \vec{J}_{23} \quad \vec{J}_{14}, \vec{J}_{24} \text{ are similar }, \\
    \vec{J}_{1}, \vec{J}_{2} \text{ are similar }\\
    \implies \vec{N}_{1}, \vec{N}_{2} \text{ are similar }
\end{align*}}
\\& Case 3: 
\\&
\parbox{12cm}{\begin{align*}
    \vec{J_1}=\myvec{\vec{J}_{11} & 0&0&0&0 \\ 0&\vec{J}_{12}&0&0&0\\0&0&\vec{J}_{13}&0&0\\0&0&0&\vec{J}_{14}&0\\0&0&0&0&\vec{J}_{15}} \quad \vec{J_2}=\myvec{\vec{J}_{21} & 0&0&0&0 \\ 0&\vec{J}_{22}&0&0&0\\0&0&\vec{J}_{23}&0&0\\0&0&0&\vec{J}_{24}&0\\0&0&0&0&\vec{J}_{25}}\\
    \vec{J}_{11}: 2\times2 \quad \vec{J}_{21}: 2\times2\\
    \vec{J}_{12}: 1\times1 \quad \vec{J}_{22}: 1\times1\\
    \vec{J}_{13}: 1\times1 \quad \vec{J}_{23}: 1\times1\\
    \vec{J}_{14}: 1\times1 \quad \vec{J}_{24}: 1\times1\\
    \vec{J}_{15}: 1\times1 \quad \vec{J}_{25}: 1\times1\\   
    \vec{J}_{11}, \vec{J}_{21}\quad \vec{J}_{12}, \vec{J}_{22}  \text{ and } \vec{J}_{13}, \vec{J}_{23} \quad \vec{J}_{14}, \vec{J}_{24} \quad \vec{J}_{15}, \vec{J}_{25} \text{ are similar }, \\
    \vec{J}_{1}, \vec{J}_{2} \text{ are similar }\\
    \implies \vec{N}_{1}, \vec{N}_{2} \text{ are similar }
\end{align*}}\\
\hline
Matrix size - 6, Jordan size-1+1+1+1+1+1 & $\vec{J_1}$ and $\vec{J_2}$ will have same number of blocks because of (3).
\\&
\parbox{12cm}{\begin{align*}
    \vec{J_1}=\myvec{\vec{J}_{11}&0&0&0&0&0 \\ 0&\vec{J}_{12}&0&0&0&0 \\ 0&0&\vec{J}_{13}&0&0&0\\ 0&0&0&\vec{J}_{14}&0&0\\ 0&0&0&0&\vec{J}_{15}&0\\ 0&0&0&0&0&\vec{J}_{16}} \quad 
    \vec{J_2}=\myvec{\vec{J}_{21}&0&0&0&0&0 \\ 0&\vec{J}_{22}&0&0&0&0 \\ 0&0&\vec{J}_{23}&0&0&0\\ 0&0&0&\vec{J}_{24}&0&0\\ 0&0&0&0&\vec{J}_{25}&0\\ 0&0&0&0&0&\vec{J}_{26}}\\
    \vec{J}_{11}, \vec{J}_{21} \quad \vec{J}_{12}, \vec{J}_{22} \quad \vec{J}_{13}, \vec{J}_{23} \quad \vec{J}_{14}, \vec{J}_{24} \quad \vec{J}_{15}, \vec{J}_{25} \text{ and } \vec{J}_{16}, \vec{J}_{26} \text{ are similar }, \\
    \vec{J}_{1}, \vec{J}_{2} \text{ are similar }\\
    \implies \vec{N}_{1}, \vec{N}_{2} \text{ are similar }
\end{align*}}\\
\hline
$7 \times 7$ Nielpotent matrix & 
Let us take a counter example,\\
& Matrix size - 7, Jordan block size - 3+3+1 and 3+2+2\\
& \parbox{10cm}{\begin{align}
    \vec{J_1} = \myvec{0&0&0&0&0&0&0\\1&0&0&0&0&0&0\\0&1&0&0&0&0&0\\0&0&0&0&0&0&0\\0&0&0&1&0&0&0\\0&0&0&0&1&0&0\\0&0&0&0&0&0&0} \quad 
    \vec{J_2} = \myvec{0&0&0&0&0&0&0\\1&0&0&0&0&0&0\\0&1&0&0&0&0&0\\0&0&0&0&0&0&0\\0&0&0&1&0&0&0\\0&0&0&0&0&0&0\\0&0&0&0&0&1&0} 
\end{align}}\\
& From above, $\vec{J_1}$ and $\vec{J_2}$ are not same.\\
&Hence $\vec{N_1}$ and $\vec{N_2}$ are not similar.\\
\hline
\caption*{Table1:Solution}
\end{longtable}
\end{document}
