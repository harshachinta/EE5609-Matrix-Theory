\documentclass[journal,12pt,twocolumn]{IEEEtran}

\usepackage{setspace}
\usepackage{gensymb}

\singlespacing


\usepackage[cmex10]{amsmath}

\usepackage{amsthm}

\usepackage{mathrsfs}
\usepackage{txfonts}
\usepackage{stfloats}
\usepackage{bm}
\usepackage{cite}
\usepackage{cases}
\usepackage{subfig}

\usepackage{longtable}
\usepackage{multirow}

\usepackage{enumitem}
\usepackage{mathtools}
\usepackage{steinmetz}
\usepackage{tikz}
\usepackage{circuitikz}
\usepackage{verbatim}
\usepackage{tfrupee}
\usepackage[breaklinks=true]{hyperref}
\usepackage{tikz,pgf}

\usepackage{tkz-euclide}

\usetikzlibrary{calc,math}
\usepackage{listings}
    \usepackage{color}                                            %%
    \usepackage{array}                                            %%
    \usepackage{longtable}                                        %%
    \usepackage{calc}                                             %%
    \usepackage{multirow}                                         %%
    \usepackage{hhline}                                           %%
    \usepackage{ifthen}                                           %%
    \usepackage{lscape}     
\usepackage{multicol}
\usepackage{chngcntr}

\DeclareMathOperator*{\Res}{Res}

\renewcommand\thesection{\arabic{section}}
\renewcommand\thesubsection{\thesection.\arabic{subsection}}
\renewcommand\thesubsubsection{\thesubsection.\arabic{subsubsection}}

\renewcommand\thesectiondis{\arabic{section}}
\renewcommand\thesubsectiondis{\thesectiondis.\arabic{subsection}}
\renewcommand\thesubsubsectiondis{\thesubsectiondis.\arabic{subsubsection}}


\hyphenation{op-tical net-works semi-conduc-tor}
\def\inputGnumericTable{}                                 %%

\lstset{
%language=C,
frame=single, 
breaklines=true,
columns=fullflexible
}
\begin{document}


\newtheorem{theorem}{Theorem}[section]
\newtheorem{problem}{Problem}
\newtheorem{proposition}{Proposition}[section]
\newtheorem{lemma}{Lemma}[section]
\newtheorem{corollary}[theorem]{Corollary}
\newtheorem{example}{Example}[section]
\newtheorem{definition}[problem]{Definition}

\newcommand{\BEQA}{\begin{eqnarray}}
\newcommand{\EEQA}{\end{eqnarray}}
\newcommand{\define}{\stackrel{\triangle}{=}}
\bibliographystyle{IEEEtran}

\providecommand{\mbf}{\mathbf}
\providecommand{\pr}[1]{\ensuremath{\Pr\left(#1\right)}}
\providecommand{\qfunc}[1]{\ensuremath{Q\left(#1\right)}}
\providecommand{\sbrak}[1]{\ensuremath{{}\left[#1\right]}}
\providecommand{\lsbrak}[1]{\ensuremath{{}\left[#1\right.}}
\providecommand{\rsbrak}[1]{\ensuremath{{}\left.#1\right]}}
\providecommand{\brak}[1]{\ensuremath{\left(#1\right)}}
\providecommand{\lbrak}[1]{\ensuremath{\left(#1\right.}}
\providecommand{\rbrak}[1]{\ensuremath{\left.#1\right)}}
\providecommand{\cbrak}[1]{\ensuremath{\left\{#1\right\}}}
\providecommand{\lcbrak}[1]{\ensuremath{\left\{#1\right.}}
\providecommand{\rcbrak}[1]{\ensuremath{\left.#1\right\}}}
\theoremstyle{remark}
\newtheorem{rem}{Remark}
\newcommand{\sgn}{\mathop{\mathrm{sgn}}}
\providecommand{\abs}[1]{\left\vert#1\right\vert}
\providecommand{\res}[1]{\Res\displaylimits_{#1}} 
\providecommand{\norm}[1]{\left\lVert#1\right\rVert}
%\providecommand{\norm}[1]{\lVert#1\rVert}
\providecommand{\mtx}[1]{\mathbf{#1}}
\providecommand{\mean}[1]{E\left[ #1 \right]}
\providecommand{\fourier}{\overset{\mathcal{F}}{ \rightleftharpoons}}
%\providecommand{\hilbert}{\overset{\mathcal{H}}{ \rightleftharpoons}}
\providecommand{\system}{\overset{\mathcal{H}}{ \longleftrightarrow}}
	%\newcommand{\solution}[2]{\textbf{Solution:}{#1}}
\newcommand{\solution}{\noindent \textbf{Solution: }}
\newcommand{\cosec}{\,\text{cosec}\,}
\providecommand{\dec}[2]{\ensuremath{\overset{#1}{\underset{#2}{\gtrless}}}}
\newcommand{\myvec}[1]{\ensuremath{\begin{pmatrix}#1\end{pmatrix}}}
\newcommand{\mydet}[1]{\ensuremath{\begin{vmatrix}#1\end{vmatrix}}}

\numberwithin{equation}{subsection}

\makeatletter
\@addtoreset{figure}{problem}
\makeatother
\let\StandardTheFigure\thefigure
\let\vec\mathbf

\renewcommand{\thefigure}{\theproblem}

\def\putbox#1#2#3{\makebox[0in][l]{\makebox[#1][l]{}\raisebox{\baselineskip}[0in][0in]{\raisebox{#2}[0in][0in]{#3}}}}
     \def\rightbox#1{\makebox[0in][r]{#1}}
     \def\centbox#1{\makebox[0in]{#1}}
     \def\topbox#1{\raisebox{-\baselineskip}[0in][0in]{#1}}
     \def\midbox#1{\raisebox{-0.5\baselineskip}[0in][0in]{#1}}
\vspace{3cm}
\title{Assignment 5}
\author{Sri Harsha CH}

\maketitle
\newpage

\bigskip
\renewcommand{\thefigure}{\theenumi}
\renewcommand{\thetable}{\theenumi}

\begin{abstract}
This document explains one of the property of triangles.
\end{abstract}
Download latex-tikz codes from 
%
\begin{lstlisting}
https://github.com/harshachinta/EE5609-Matrix-Theory/tree/master/Assignments/Assignment3
\end{lstlisting}
%
\section{Problem}
The line-segment joining the mid-points of any two sides of a triangle is parallel to the third side and is half of it.
\section{Explanation}
Let us consider a $\triangle{ABC}$, and let $\vec{D}$,$\vec{E}$ and $\vec{F}$ be the mid- points of sides AB,BC and CA respectively.\\
Let us consider a line-segment joining the points $\vec{D}$ and $\vec{F}$ which are midpoints of line AB and CA.\\
As $\vec{D}$ is midpoint of line AB, $\vec{E}$ is midpoint of line BC and $\vec{F}$ is midpoint of line CA, they can be written as follows:
\begin{align}
    \vec{D}&=\frac{A+B}{2} \label{eq:mid_1}\\
    \vec{E}&=\frac{B+C}{2} \label{eq:mid_2}\\
    \vec{F}&=\frac{C+A}{2} \label{eq:mid_3}
\end{align}
The line DF can be written in the form of direction vector as,
\begin{align}
    \vec{m}_{DF}&=\vec{D}-\vec{F} \label{eq:dir_1}
\end{align}
Substituting equation \eqref{eq:mid_1} and \eqref{eq:mid_3} in \eqref{eq:dir_1} we get,
\begin{align}
    \vec{m}_{DF}&=\frac{A+B}{2}-\frac{C+A}{2}\\
    \vec{m}_{DF}&=\frac{B-C}{2} \\
    \vec{m}_{DF}&=\frac{\vec{m}_{BC}}{2} \label{eq:res_1}
\end{align}
where $\vec{m}_{BC}$ is the direction vector of line BC.\\
From equation \eqref{eq:res_1} we can say that the direction vectors $\vec{m}_{DF}$ and $\vec{m}_{BC}$ are in same direction. Thus, line-segment DF joining the mid-points of two sides (AB and AC) of triangle is parallel to the third side (BC) of the triangle and is half of it.\\
\\
This can be applied to other 2 sides as well,
The line DE joining the mid-points D and E can be written in form of direction vector as:
\begin{align}
    \vec{m}_{DE}&=\vec{D}-\vec{E} \label{eq:dir_2}
\end{align}


Substituting equation \eqref{eq:mid_1} and \eqref{eq:mid_2} in \eqref{eq:dir_2} we get,
\begin{align}
    \vec{m}_{DE}&=\frac{A+B}{2}-\frac{B+C}{2}\\
    \vec{m}_{DE}&=\frac{A-C}{2} \\
    \vec{m}_{DE}&=\frac{\vec{m}_{AC}}{2} \label{eq:res_2}
\end{align}
where $\vec{m}_{AC}$ is the direction vector of line AC.\\
Similarly, The line EF joining the mid-points E and F can be written in form of direction vector as:
\begin{align}
    \vec{m}_{EF}&=\vec{E}-\vec{F} \label{eq:dir_3}
\end{align}
Substituting equation \eqref{eq:mid_2} and \eqref{eq:mid_3} in \eqref{eq:dir_3} we get,
\begin{align}
    \vec{m}_{EF}&=\frac{B+C}{2}-\frac{C+A}{2}\\
    \vec{m}_{EF}&=\frac{B-A}{2} \\
    \vec{m}_{EF}&=\frac{\vec{m}_{BA}}{2} \label{eq:res_3}
\end{align}
where $\vec{m}_{BA}$ is the direction vector of line BA.\\

\section{Solution}
From equations \eqref{eq:res_1}, \eqref{eq:res_2} and \eqref{eq:res_3}, we can say that the line-segment joining the mid-points of any two sides of a triangle is parallel to the third side and is half of it
\\\end{document}
