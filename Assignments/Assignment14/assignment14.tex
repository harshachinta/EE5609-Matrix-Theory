\documentclass[journal,12pt,twocolumn]{IEEEtran}

\usepackage{setspace}
\usepackage{gensymb}

\singlespacing


\usepackage[cmex10]{amsmath}

\usepackage{amsthm}

\usepackage{mathrsfs}
\usepackage{txfonts}
\usepackage{stfloats}
\usepackage{bm}
\usepackage{cite}
\usepackage{cases}
\usepackage{subfig}

\usepackage{longtable}
\usepackage{multirow}

\usepackage{enumitem}
\usepackage{mathtools}
\usepackage{steinmetz}
\usepackage{tikz}
\usepackage{circuitikz}
\usepackage{verbatim}
\usepackage{tfrupee}
\usepackage[breaklinks=true]{hyperref}

\usepackage{tkz-euclide}

\usetikzlibrary{calc,math}
\usepackage{listings}
    \usepackage{color}                                            %%
    \usepackage{array}                                            %%
    \usepackage{longtable}                                        %%
    \usepackage{calc}                                             %%
    \usepackage{multirow}                                         %%
    \usepackage{hhline}                                           %%
    \usepackage{ifthen}                                           %%
    \usepackage{lscape}     
\usepackage{multicol}
\usepackage{chngcntr}

\DeclareMathOperator*{\Res}{Res}

\renewcommand\thesection{\arabic{section}}
\renewcommand\thesubsection{\thesection.\arabic{subsection}}
\renewcommand\thesubsubsection{\thesubsection.\arabic{subsubsection}}

\renewcommand\thesectiondis{\arabic{section}}
\renewcommand\thesubsectiondis{\thesectiondis.\arabic{subsection}}
\renewcommand\thesubsubsectiondis{\thesubsectiondis.\arabic{subsubsection}}


\hyphenation{op-tical net-works semi-conduc-tor}
\def\inputGnumericTable{}                                 %%

\lstset{
%language=C,
frame=single, 
breaklines=true,
columns=fullflexible
}
\begin{document}


\newtheorem{theorem}{Theorem}[section]
\newtheorem{problem}{Problem}
\newtheorem{proposition}{Proposition}[section]
\newtheorem{lemma}{Lemma}[section]
\newtheorem{corollary}[theorem]{Corollary}
\newtheorem{example}{Example}[section]
\newtheorem{definition}[problem]{Definition}

\newcommand{\BEQA}{\begin{eqnarray}}
\newcommand{\EEQA}{\end{eqnarray}}
\newcommand{\define}{\stackrel{\triangle}{=}}
\bibliographystyle{IEEEtran}

\providecommand{\mbf}{\mathbf}
\providecommand{\pr}[1]{\ensuremath{\Pr\left(#1\right)}}
\providecommand{\qfunc}[1]{\ensuremath{Q\left(#1\right)}}
\providecommand{\sbrak}[1]{\ensuremath{{}\left[#1\right]}}
\providecommand{\lsbrak}[1]{\ensuremath{{}\left[#1\right.}}
\providecommand{\rsbrak}[1]{\ensuremath{{}\left.#1\right]}}
\providecommand{\brak}[1]{\ensuremath{\left(#1\right)}}
\providecommand{\lbrak}[1]{\ensuremath{\left(#1\right.}}
\providecommand{\rbrak}[1]{\ensuremath{\left.#1\right)}}
\providecommand{\cbrak}[1]{\ensuremath{\left\{#1\right\}}}
\providecommand{\lcbrak}[1]{\ensuremath{\left\{#1\right.}}
\providecommand{\rcbrak}[1]{\ensuremath{\left.#1\right\}}}
\theoremstyle{remark}
\newtheorem{rem}{Remark}
\newcommand{\sgn}{\mathop{\mathrm{sgn}}}
\providecommand{\abs}[1]{\left\vert#1\right\vert}
\providecommand{\res}[1]{\Res\displaylimits_{#1}} 
\providecommand{\norm}[1]{\left\lVert#1\right\rVert}
%\providecommand{\norm}[1]{\lVert#1\rVert}
\providecommand{\mtx}[1]{\mathbf{#1}}
\providecommand{\mean}[1]{E\left[ #1 \right]}
\providecommand{\fourier}{\overset{\mathcal{F}}{ \rightleftharpoons}}
%\providecommand{\hilbert}{\overset{\mathcal{H}}{ \rightleftharpoons}}
\providecommand{\system}{\overset{\mathcal{H}}{ \longleftrightarrow}}
	%\newcommand{\solution}[2]{\textbf{Solution:}{#1}}
\newcommand{\solution}{\noindent \textbf{Solution: }}
\newcommand{\cosec}{\,\text{cosec}\,}
\providecommand{\dec}[2]{\ensuremath{\overset{#1}{\underset{#2}{\gtrless}}}}
\newcommand{\myvec}[1]{\ensuremath{\begin{pmatrix}#1\end{pmatrix}}}
\newcommand{\mydet}[1]{\ensuremath{\begin{vmatrix}#1\end{vmatrix}}}

\numberwithin{equation}{subsection}

\makeatletter
\@addtoreset{figure}{problem}
\makeatother
\let\StandardTheFigure\thefigure
\let\vec\mathbf

\renewcommand{\thefigure}{\theproblem}

\def\putbox#1#2#3{\makebox[0in][l]{\makebox[#1][l]{}\raisebox{\baselineskip}[0in][0in]{\raisebox{#2}[0in][0in]{#3}}}}
     \def\rightbox#1{\makebox[0in][r]{#1}}
     \def\centbox#1{\makebox[0in]{#1}}
     \def\topbox#1{\raisebox{-\baselineskip}[0in][0in]{#1}}
     \def\midbox#1{\raisebox{-0.5\baselineskip}[0in][0in]{#1}}
\vspace{3cm}
\title{Assignment 14}
\author{Sri Harsha CH}

\maketitle
\newpage

\bigskip
\renewcommand{\thefigure}{\theenumi}
\renewcommand{\thetable}{\theenumi}

\begin{abstract}
This document explains the conditions to check for a sub space. 
\end{abstract}

Download all python codes from 
\begin{lstlisting}
https://github.com/harshachinta/EE5609-Matrix-Theory/tree/master/Assignments/Assignment14/code
\end{lstlisting}
%
and latex-tikz codes from 
%
\begin{lstlisting}
https://github.com/harshachinta/EE5609-Matrix-Theory/tree/master/Assignments/Assignment14
\end{lstlisting}
%
\section{Problem}
Let $\vec{V}$ be the  vector space of all functions from $\vec{R}$ into $\vec{R}$; let $\vec{V_e}$ be the subset of even functions, $f(-x) = f(x)$; let $\vec{V_o}$ be the subset of odd functions, $f(-x) = -f(x)$.
\begin{enumerate}
   \item Prove that $\vec{V_e}$ and $\vec{V_o}$ are subspaces of $\vec{V}$
   \item Prove that $\vec{V_e} + \vec{V_o} = \vec{V}$
    \item Prove that $\vec{V_e} \cap \vec{V_o} = \{0\}$
 \end{enumerate}
\section{Explanation}
 \begin{enumerate}
   \item Prove that $\vec{V_e}$ and $\vec{V_o}$ are subspaces of $\vec{V}$.
 \end{enumerate}

A non-empty subset $\vec{W}$ of $\vec{V}$ is a subspace of $\vec{V}$ if and only if for each pair of vectors $\vec{\alpha}$, $\vec{\beta}$ in $\vec{W}$ and each scalar $c$ in $\vec{F}$ the vector $c\vec{\alpha}+\vec{\beta}$ is again in $\vec{W}$.\\

Let $\vec{u}$, $\vec{v}$ $\in$ $\vec{V_e}$ and $c$ $\in$ $\vec{R}$ and let $\vec{h}$ = $c\vec{u}$ + $\vec{v}$. Then,
\begin{equation} \label{eq:eq1}
\begin{split}
\vec{h}(-x) &= c \vec{u}(-x) + \vec{v}(-x)\\
 & = c \vec{u}(x) + \vec{v}(x)\\
& = \vec{h}(x)\\
\end{split}
\end{equation}
From \eqref{eq:eq1}
\begin{align}
\implies \vec{h}(-x) = \vec{h}(x)\\
\implies \vec{h} \in \vec{V_e} \label{eq:eq2}
\end{align}

Let $\vec{u}$, $\vec{v}$ $\in$ $\vec{V_o}$ and $c$ $\in$ $\vec{R}$ and let $\vec{h}$ = $c\vec{u}$ + $\vec{v}$. Then,
\begin{equation} \label{eq:eq3}
\begin{split}
\vec{h}(-x) &= c \vec{u}(-x) + \vec{v}(-x)\\
 & = -c \vec{u}(x) - \vec{v}(x)\\
& = -\vec{h}(x)\\
\end{split}
\end{equation}
From \eqref{eq:eq3}
\begin{align}
\implies \vec{h}(-x) = -\vec{h}(x)\\
\implies \vec{h} \in \vec{V_o} \label{eq:eq4}
\end{align}
From \eqref{eq:eq2} and \eqref{eq:eq4}, $\vec{V_e}$ and $\vec{V_o}$ are subspaces of $\vec{V}$.\\
\begin{enumerate}
    \setcounter{enumi}{1}
   \item Prove that $\vec{V_e} + \vec{V_o} = \vec{V}$.
 \end{enumerate}
Let $\vec{u}$ $\in$ $\vec{V}$
\begin{align}
    \vec{u_e}(x) = \frac{\vec{u}(x)+\vec{u}(-x)}{2} \label{eq:eq10}\\
    \vec{u_o}(x) = \frac{\vec{u}(x)-\vec{u}(-x)}{2} \label{eq:eq11}
\end{align}
Equation equation \eqref{eq:eq10} and \eqref{eq:eq11}, $\vec{u_e}$ is even and $\vec{u_o}$ is odd. Adding both the equations,
\begin{align}
    \vec{u} = \vec{u_e} + \vec{u_o} \label{eq:eq12}
\end{align}

\begin{enumerate}
    \setcounter{enumi}{2}
   \item Prove that $\vec{V_e} \cap \vec{V_o} = \{0\}$.
 \end{enumerate}
Let $\vec{u} \in \vec{V_e} \cap \vec{V_o}$
\begin{align}
    \vec{u} \in \vec{V_e} \implies \vec{u}(-x) &= \vec{u}(x) \label{eq:eq7}\\
    \vec{u} \in \vec{V_o} \implies \vec{u}(-x) &= -\vec{u}(x) \label{eq:eq8}
\end{align}
Equating \eqref{eq:eq7} and \eqref{eq:eq8},
\begin{align}
    \vec{u}(x) = -\vec{u}(x)\\
    \implies 2\vec{u}(x)=0\\
    \implies \vec{u}=0 \label{eq:eq13}
\end{align}
\section{Solution}
Equations \eqref{eq:eq2}, \eqref{eq:eq4}, \eqref{eq:eq12}, \eqref{eq:eq13} proves 1, 2 and 3.
\\\end{document}
