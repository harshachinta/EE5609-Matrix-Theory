\documentclass[journal,12pt,twocolumn]{IEEEtran}

\usepackage{setspace}
\usepackage{gensymb}

\singlespacing


\usepackage[cmex10]{amsmath}

\usepackage{amsthm}

\usepackage{mathrsfs}
\usepackage{txfonts}
\usepackage{stfloats}
\usepackage{bm}
\usepackage{cite}
\usepackage{cases}
\usepackage{subfig}

\usepackage{longtable}
\usepackage{multirow}

\usepackage{enumitem}
\usepackage{mathtools}
\usepackage{steinmetz}
\usepackage{tikz}
\usepackage{circuitikz}
\usepackage{verbatim}
\usepackage{tfrupee}
\usepackage[breaklinks=true]{hyperref}

\usepackage{tkz-euclide}

\usetikzlibrary{calc,math}
\usepackage{listings}
    \usepackage{color}                                            %%
    \usepackage{array}                                            %%
    \usepackage{longtable}                                        %%
    \usepackage{calc}                                             %%
    \usepackage{multirow}                                         %%
    \usepackage{hhline}                                           %%
    \usepackage{ifthen}                                           %%
    \usepackage{lscape}     
\usepackage{multicol}
\usepackage{chngcntr}

\DeclareMathOperator*{\Res}{Res}

\renewcommand\thesection{\arabic{section}}
\renewcommand\thesubsection{\thesection.\arabic{subsection}}
\renewcommand\thesubsubsection{\thesubsection.\arabic{subsubsection}}

\renewcommand\thesectiondis{\arabic{section}}
\renewcommand\thesubsectiondis{\thesectiondis.\arabic{subsection}}
\renewcommand\thesubsubsectiondis{\thesubsectiondis.\arabic{subsubsection}}


\hyphenation{op-tical net-works semi-conduc-tor}
\def\inputGnumericTable{}                                 %%

\lstset{
%language=C,
frame=single, 
breaklines=true,
columns=fullflexible
}
\begin{document}


\newtheorem{theorem}{Theorem}[section]
\newtheorem{problem}{Problem}
\newtheorem{proposition}{Proposition}[section]
\newtheorem{lemma}{Lemma}[section]
\newtheorem{corollary}[theorem]{Corollary}
\newtheorem{example}{Example}[section]
\newtheorem{definition}[problem]{Definition}

\newcommand{\BEQA}{\begin{eqnarray}}
\newcommand{\EEQA}{\end{eqnarray}}
\newcommand{\define}{\stackrel{\triangle}{=}}
\bibliographystyle{IEEEtran}

\providecommand{\mbf}{\mathbf}
\providecommand{\pr}[1]{\ensuremath{\Pr\left(#1\right)}}
\providecommand{\qfunc}[1]{\ensuremath{Q\left(#1\right)}}
\providecommand{\sbrak}[1]{\ensuremath{{}\left[#1\right]}}
\providecommand{\lsbrak}[1]{\ensuremath{{}\left[#1\right.}}
\providecommand{\rsbrak}[1]{\ensuremath{{}\left.#1\right]}}
\providecommand{\brak}[1]{\ensuremath{\left(#1\right)}}
\providecommand{\lbrak}[1]{\ensuremath{\left(#1\right.}}
\providecommand{\rbrak}[1]{\ensuremath{\left.#1\right)}}
\providecommand{\cbrak}[1]{\ensuremath{\left\{#1\right\}}}
\providecommand{\lcbrak}[1]{\ensuremath{\left\{#1\right.}}
\providecommand{\rcbrak}[1]{\ensuremath{\left.#1\right\}}}
\theoremstyle{remark}
\newtheorem{rem}{Remark}
\newcommand{\sgn}{\mathop{\mathrm{sgn}}}
\providecommand{\abs}[1]{\left\vert#1\right\vert}
\providecommand{\res}[1]{\Res\displaylimits_{#1}} 
\providecommand{\norm}[1]{\left\lVert#1\right\rVert}
%\providecommand{\norm}[1]{\lVert#1\rVert}
\providecommand{\mtx}[1]{\mathbf{#1}}
\providecommand{\mean}[1]{E\left[ #1 \right]}
\providecommand{\fourier}{\overset{\mathcal{F}}{ \rightleftharpoons}}
%\providecommand{\hilbert}{\overset{\mathcal{H}}{ \rightleftharpoons}}
\providecommand{\system}{\overset{\mathcal{H}}{ \longleftrightarrow}}
	%\newcommand{\solution}[2]{\textbf{Solution:}{#1}}
\newcommand{\solution}{\noindent \textbf{Solution: }}
\newcommand{\cosec}{\,\text{cosec}\,}
\providecommand{\dec}[2]{\ensuremath{\overset{#1}{\underset{#2}{\gtrless}}}}
\newcommand{\myvec}[1]{\ensuremath{\begin{pmatrix}#1\end{pmatrix}}}
\newcommand{\mydet}[1]{\ensuremath{\begin{vmatrix}#1\end{vmatrix}}}

\numberwithin{equation}{subsection}

\makeatletter
\@addtoreset{figure}{problem}
\makeatother
\let\StandardTheFigure\thefigure
\let\vec\mathbf

\renewcommand{\thefigure}{\theproblem}

\def\putbox#1#2#3{\makebox[0in][l]{\makebox[#1][l]{}\raisebox{\baselineskip}[0in][0in]{\raisebox{#2}[0in][0in]{#3}}}}
     \def\rightbox#1{\makebox[0in][r]{#1}}
     \def\centbox#1{\makebox[0in]{#1}}
     \def\topbox#1{\raisebox{-\baselineskip}[0in][0in]{#1}}
     \def\midbox#1{\raisebox{-0.5\baselineskip}[0in][0in]{#1}}
\vspace{3cm}
\title{Assignment 16}
\author{Sri Harsha CH}

\maketitle
\newpage

\bigskip
\renewcommand{\thefigure}{\theenumi}
\renewcommand{\thetable}{\theenumi}

\begin{abstract}
This document explains the conditions for two vector spaces to be isomorphic. 
\end{abstract}

Download all python codes from 
\begin{lstlisting}
https://github.com/harshachinta/EE5609-Matrix-Theory/tree/master/Assignments/Assignment16/code
\end{lstlisting}
%
and latex-tikz codes from 
%
\begin{lstlisting}
https://github.com/harshachinta/EE5609-Matrix-Theory/tree/master/Assignments/Assignment16
\end{lstlisting}
%
\section{Problem}
Let $\vec{V}$ and $\vec{W}$ be finite-dimensional vector spaces over the field $\vec{F}$. Prove that $\vec{V}$ and $\vec{W}$ are isomorphic if and only if $\dim{\vec{V}}$ = $\dim{\vec{W}}$
\section{Explanation}
If $\vec{V}$ and $\vec{W}$ are vector spaces over the field $\vec{F}$, any one to one linear transformation T of $\vec{V}$ onto $\vec{W}$ is called an isomorphism of $\vec{V}$ onto $\vec{W}$.\\

Let T : $\vec{V}$ $\xrightarrow{}$ $\vec{W}$ be an isomorphism and let $\{\vec{v_1}, \vec{v_2},\cdots,\vec{v_n}\}$ be a basis for $\vec{V}$. Since it is isomorphic we know that T is one to one and onto. We need to show that  $\{T(\vec{v_1}), T(\vec{v_2}),\cdots,T(\vec{v_n})\}$ is a basis for $\vec{W}$ so that $\dim{\vec{V}}$ = n = $\dim{\vec{W}}$.\\






\begin{table*}[ht!]
\begin{center}
\begin{tabular}{|l|l|}
\hline
\multicolumn{2}{|c|}{$\vec{V} = \{\vec{v_1}, \vec{v_2},\cdots,\vec{v_n}\}$ \quad $\vec{W} = \{T(\vec{v_1}), T(\vec{v_2}),\cdots,T(\vec{v_n})\}$}\\[1ex]
\hline
\textbf{Property Used} & \textbf{Derivation} \\[0.5ex]
\hline
\text{T is one-one} & 
\text{Linear combination of vectors in $\vec{W}$}\\
& \parbox{10cm}{\begin{align}
    \sum_{k=1}^{n} \alpha_kT(\vec{v_k}) = \vec{0} \label{eq:eq1}\\
    \sum_{k=1}^{n} T(\alpha_k\vec{v_k}) = \vec{0}\\
    \implies \sum_{k=1}^{n} \alpha_k\vec{v_k} = \vec{0}\\
    \implies \alpha_1=\alpha_2=\cdots=\alpha_n=0 \label{eq:eq2}
\end{align}} \\
& From equation \eqref{eq:eq1} and \eqref{eq:eq2}, the set of vectors $\{T(\vec{v_1}), T(\vec{v_2}),\cdots,T(\vec{v_n})\}$ \\ &are linearly independent
\\ [0.5ex] 
\hline
\text{T is onto} & 
\text{For any $\vec{y} \in \vec{W}$ there exists an $\vec{x} \in \vec{V}$ such that $T(\vec{x}) = \vec{y}$.}\\
& \parbox{10cm}{\begin{align}
    \vec{x} = \sum_{k=1}^{n} \alpha_k\vec{v_k}\\
    T(\vec{x}) = T(\sum_{k=1}^{n} \alpha_k\vec{v_k} ) = \vec{y}\\
    \sum_{k=1}^{n} \alpha_kT(\vec{v_k}) = \vec{y} \label{eq:eq3}
\end{align}} \\
& From equation \eqref{eq:eq3}, any vector in $\vec{W}$ can be represented as linear combination\\ & of $\{T(\vec{v_1}), T(\vec{v_2}),\cdots,T(\vec{v_n})\}$. That is it spans $\vec{W}$.
\\ [0.5ex] 
\hline
\end{tabular}
\caption{Derivation}
\label{table:1}
\end{center}
\vspace{-0.5cm}
\end{table*}







Therefore from equation \eqref{eq:eq2} and \eqref{eq:eq3} in Table \ref{table:1}, $\{T(\vec{v_1}), T(\vec{v_2}),\cdots,T(\vec{v_n})\}$ are linearly independent and span $\vec{W}$.\\

$\implies \dim{\vec{W}}$ = n = $\dim{\vec{V}}$














\section{Solution}
$\vec{V}$ and $\vec{W}$ are isomorphic if and only if $\dim{\vec{V}}$ = $\dim{\vec{W}}$
\\\end{document}
