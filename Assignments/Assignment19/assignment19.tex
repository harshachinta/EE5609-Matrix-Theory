\documentclass[journal,12pt,twocolumn]{IEEEtran}

\usepackage{setspace}
\usepackage{gensymb}

\singlespacing


\usepackage[cmex10]{amsmath}

\usepackage{amsthm}

\usepackage{mathrsfs}
\usepackage{txfonts}
\usepackage{stfloats}
\usepackage{bm}
\usepackage{cite}
\usepackage{cases}
\usepackage{subfig}

\usepackage{longtable}
\usepackage{multirow}

\usepackage{enumitem}
\usepackage{mathtools}
\usepackage{steinmetz}
\usepackage{tikz}
\usepackage{circuitikz}
\usepackage{verbatim}
\usepackage{tfrupee}
\usepackage[breaklinks=true]{hyperref}

\usepackage{tkz-euclide}

\usetikzlibrary{calc,math}
\usepackage{listings}
    \usepackage{color}                                            %%
    \usepackage{array}                                            %%
    \usepackage{longtable}                                        %%
    \usepackage{calc}                                             %%
    \usepackage{multirow}                                         %%
    \usepackage{hhline}                                           %%
    \usepackage{ifthen}                                           %%
    \usepackage{lscape}     
\usepackage{multicol}
\usepackage{chngcntr}

\DeclareMathOperator*{\Res}{Res}
\DeclareMathOperator{\range}{range}

\renewcommand\thesection{\arabic{section}}
\renewcommand\thesubsection{\thesection.\arabic{subsection}}
\renewcommand\thesubsubsection{\thesubsection.\arabic{subsubsection}}

\renewcommand\thesectiondis{\arabic{section}}
\renewcommand\thesubsectiondis{\thesectiondis.\arabic{subsection}}
\renewcommand\thesubsubsectiondis{\thesubsectiondis.\arabic{subsubsection}}


\hyphenation{op-tical net-works semi-conduc-tor}
\def\inputGnumericTable{}                                 %%

\lstset{
%language=C,
frame=single, 
breaklines=true,
columns=fullflexible
}
\begin{document}


\newtheorem{theorem}{Theorem}[section]
\newtheorem{problem}{Problem}
\newtheorem{proposition}{Proposition}[section]
\newtheorem{lemma}{Lemma}[section]
\newtheorem{corollary}[theorem]{Corollary}
\newtheorem{example}{Example}[section]
\newtheorem{definition}[problem]{Definition}

\newcommand{\BEQA}{\begin{eqnarray}}
\newcommand{\EEQA}{\end{eqnarray}}
\newcommand{\define}{\stackrel{\triangle}{=}}
\bibliographystyle{IEEEtran}

\providecommand{\mbf}{\mathbf}
\providecommand{\pr}[1]{\ensuremath{\Pr\left(#1\right)}}
\providecommand{\qfunc}[1]{\ensuremath{Q\left(#1\right)}}
\providecommand{\sbrak}[1]{\ensuremath{{}\left[#1\right]}}
\providecommand{\lsbrak}[1]{\ensuremath{{}\left[#1\right.}}
\providecommand{\rsbrak}[1]{\ensuremath{{}\left.#1\right]}}
\providecommand{\brak}[1]{\ensuremath{\left(#1\right)}}
\providecommand{\lbrak}[1]{\ensuremath{\left(#1\right.}}
\providecommand{\rbrak}[1]{\ensuremath{\left.#1\right)}}
\providecommand{\cbrak}[1]{\ensuremath{\left\{#1\right\}}}
\providecommand{\lcbrak}[1]{\ensuremath{\left\{#1\right.}}
\providecommand{\rcbrak}[1]{\ensuremath{\left.#1\right\}}}
\theoremstyle{remark}
\newtheorem{rem}{Remark}
\newcommand{\sgn}{\mathop{\mathrm{sgn}}}
\providecommand{\abs}[1]{\left\vert#1\right\vert}
\providecommand{\res}[1]{\Res\displaylimits_{#1}} 
\providecommand{\norm}[1]{\left\lVert#1\right\rVert}
%\providecommand{\norm}[1]{\lVert#1\rVert}
\providecommand{\mtx}[1]{\mathbf{#1}}
\providecommand{\mean}[1]{E\left[ #1 \right]}
\providecommand{\fourier}{\overset{\mathcal{F}}{ \rightleftharpoons}}
%\providecommand{\hilbert}{\overset{\mathcal{H}}{ \rightleftharpoons}}
\providecommand{\system}{\overset{\mathcal{H}}{ \longleftrightarrow}}
	%\newcommand{\solution}[2]{\textbf{Solution:}{#1}}
\newcommand{\solution}{\noindent \textbf{Solution: }}
\newcommand{\cosec}{\,\text{cosec}\,}
\providecommand{\dec}[2]{\ensuremath{\overset{#1}{\underset{#2}{\gtrless}}}}
\newcommand{\myvec}[1]{\ensuremath{\begin{pmatrix}#1\end{pmatrix}}}
\newcommand{\mydet}[1]{\ensuremath{\begin{vmatrix}#1\end{vmatrix}}}

\numberwithin{equation}{subsection}

\makeatletter
\@addtoreset{figure}{problem}
\makeatother
\let\StandardTheFigure\thefigure
\let\vec\mathbf

\renewcommand{\thefigure}{\theproblem}

\def\putbox#1#2#3{\makebox[0in][l]{\makebox[#1][l]{}\raisebox{\baselineskip}[0in][0in]{\raisebox{#2}[0in][0in]{#3}}}}
     \def\rightbox#1{\makebox[0in][r]{#1}}
     \def\centbox#1{\makebox[0in]{#1}}
     \def\topbox#1{\raisebox{-\baselineskip}[0in][0in]{#1}}
     \def\midbox#1{\raisebox{-0.5\baselineskip}[0in][0in]{#1}}
\vspace{3cm}
\title{Assignment 19}
\author{Sri Harsha CH}

\maketitle
\newpage

\bigskip
\renewcommand{\thefigure}{\theenumi}
\renewcommand{\thetable}{\theenumi}

\begin{abstract}
This document explains the representation of transformations by matrix.
\end{abstract}

Download all python codes from 
\begin{lstlisting}
https://github.com/harshachinta/EE5609-Matrix-Theory/tree/master/Assignments/Assignment19/code
\end{lstlisting}
%
and latex-tikz codes from 
%
\begin{lstlisting}
https://github.com/harshachinta/EE5609-Matrix-Theory/tree/master/Assignments/Assignment19
\end{lstlisting}
%
\section{Problem}
Let $\vec{A}$ be an $m \times n$ matrix of rank $m$ with $n>m$. If for some non-zero real number $\alpha$, we have $\vec{x^TAA^Tx} = \alpha\vec{x^Tx}$, for all $x \in \vec{R^m}$, then $\vec{A^TA}$ has,

1. exactly two distinct eigenvalues.

2. 0 as an eigenvalue with multiplicity $n-m$.

3. $\alpha$ as a non-zero eigenvalue.

4. exactly two non-zero distinct eigenvalues.

\section{Explanation}
Refer Table \ref{table:1}.


\begin{table*}[ht!]
\begin{center}
\begin{tabular}{|l|l|}
\hline
\textbf{Given} & \textbf{Derivation} \\[0.5ex]
\hline
\text{Given} & 
\text{$\vec{A}$ is a $m \times n$ matrix of rank $m$ with $n>m$}. \\
& A non-zero real number $\alpha$.\\
& To find eigenvalues of $\vec{A^TA}$.
\\ [0.5ex]
\hline
\text{Eigenvalues of $\vec{AA^T}$} & 
\text{$\vec{AA^T}$ is a $m \times m$ matrix and $\vec{A^TA}$ is a $n \times n$ matrix.}\\
& Let, $\lambda$ be a non-zero eigen value of $\vec{A^TA}$.\\
& \parbox{10cm}{\begin{align}
    \vec{A^TAv} = \lambda \vec{v} \quad \vec{v} \in \vec{R^n}\\
    \vec{AA^TAv} = \lambda \vec{Av}\\
    \text{Let,} \quad \vec{x} = \vec{Av} \quad \vec{x} \in \vec{R^m}\\
    \vec{AA^Tx} = \lambda \vec{x}\\
    \vec{x^TAA^Tx} = \lambda \vec{x^Tx}\\
    \text{Given}, \quad \vec{x^TAA^Tx} = \alpha\vec{x^Tx}\\
    \implies \alpha\vec{x^Tx} = \lambda \vec{x^Tx} \label{eq:eq18}
\end{align}} \\
& From equation \eqref{eq:eq18}, $\lambda = \alpha$ as $\norm{\vec{x}} \not = 0$\\
& $\vec{A^TA}$ has an eigenvalue $\alpha$ with multiplicity $m$.
\\ [0.5ex]
\hline
\text{Eigenvalues of $\vec{A^TA}$} & 
$\vec{A^TA}$ is a $n \times n$ matrix. Given $n > m$, \\\\
&We know that, $\vec{A^TA}$ and $\vec{AA^T}$ have same number of non-zero eigenvalues\\& and if one of them has more number of eigenvalues than the other \\&then these eigenvalues are zero.\\
& 1. From above, as $\alpha$ is non-zero, $\vec{A^TA}$ has $\alpha$ as its eigenvalue with multiplicity $m$ \\
& 2. $\vec{A^TA}$ has $0$ as its eigenvalue with multiplicity $n-m$\\
& 3. Therefore, the two distinct eigenvalues of $\vec{A^TA}$ are $\alpha$ and $0$.
\\ [0.5ex]
\hline
\end{tabular}
\caption{Explanation}
\label{table:1}
\end{center}
\vspace{-0.5cm}
\end{table*}

\section{Solution}
Refer Table \ref{table:2}.

\renewcommand{\thetable}{1}
\begin{table*}[ht]
\begin{center}
\begin{tabular}{|c|c|}
\hline
& \\
$\vec{A^TA}$ has exactly two distinct eigenvalues.
& True statement\\
\hline
& \\
$\vec{A^TA}$ has 0 as an eigenvalue with multiplicity $n-m$
& True statement\\
\hline
& \\
$\vec{A^TA}$ has $\alpha$ as a non-zero eigenvalue
&  True statement\\
\hline
& \\
$\vec{A^TA}$ has exactly two non-zero distinct eigenvalues.
& False statement\\
\hline
\end{tabular}
\caption{Solution}
\label{table:2}
\end{center}
\end{table*}
\end{document}
