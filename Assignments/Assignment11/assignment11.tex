\documentclass[journal,12pt,twocolumn]{IEEEtran}

\usepackage{setspace}
\usepackage{gensymb}

\singlespacing


\usepackage[cmex10]{amsmath}

\usepackage{amsthm}

\usepackage{mathrsfs}
\usepackage{txfonts}
\usepackage{stfloats}
\usepackage{bm}
\usepackage{cite}
\usepackage{cases}
\usepackage{subfig}

\usepackage{longtable}
\usepackage{multirow}

\usepackage{enumitem}
\usepackage{mathtools}
\usepackage{steinmetz}
\usepackage{tikz}
\usepackage{circuitikz}
\usepackage{verbatim}
\usepackage{tfrupee}
\usepackage[breaklinks=true]{hyperref}

\usepackage{tkz-euclide}

\usetikzlibrary{calc,math}
\usepackage{listings}
    \usepackage{color}                                            %%
    \usepackage{array}                                            %%
    \usepackage{longtable}                                        %%
    \usepackage{calc}                                             %%
    \usepackage{multirow}                                         %%
    \usepackage{hhline}                                           %%
    \usepackage{ifthen}                                           %%
    \usepackage{lscape}     
\usepackage{multicol}
\usepackage{chngcntr}

\DeclareMathOperator*{\Res}{Res}

\renewcommand\thesection{\arabic{section}}
\renewcommand\thesubsection{\thesection.\arabic{subsection}}
\renewcommand\thesubsubsection{\thesubsection.\arabic{subsubsection}}

\renewcommand\thesectiondis{\arabic{section}}
\renewcommand\thesubsectiondis{\thesectiondis.\arabic{subsection}}
\renewcommand\thesubsubsectiondis{\thesubsectiondis.\arabic{subsubsection}}


\hyphenation{op-tical net-works semi-conduc-tor}
\def\inputGnumericTable{}                                 %%

\lstset{
%language=C,
frame=single, 
breaklines=true,
columns=fullflexible
}
\begin{document}


\newtheorem{theorem}{Theorem}[section]
\newtheorem{problem}{Problem}
\newtheorem{proposition}{Proposition}[section]
\newtheorem{lemma}{Lemma}[section]
\newtheorem{corollary}[theorem]{Corollary}
\newtheorem{example}{Example}[section]
\newtheorem{definition}[problem]{Definition}

\newcommand{\BEQA}{\begin{eqnarray}}
\newcommand{\EEQA}{\end{eqnarray}}
\newcommand{\define}{\stackrel{\triangle}{=}}
\bibliographystyle{IEEEtran}

\providecommand{\mbf}{\mathbf}
\providecommand{\pr}[1]{\ensuremath{\Pr\left(#1\right)}}
\providecommand{\qfunc}[1]{\ensuremath{Q\left(#1\right)}}
\providecommand{\sbrak}[1]{\ensuremath{{}\left[#1\right]}}
\providecommand{\lsbrak}[1]{\ensuremath{{}\left[#1\right.}}
\providecommand{\rsbrak}[1]{\ensuremath{{}\left.#1\right]}}
\providecommand{\brak}[1]{\ensuremath{\left(#1\right)}}
\providecommand{\lbrak}[1]{\ensuremath{\left(#1\right.}}
\providecommand{\rbrak}[1]{\ensuremath{\left.#1\right)}}
\providecommand{\cbrak}[1]{\ensuremath{\left\{#1\right\}}}
\providecommand{\lcbrak}[1]{\ensuremath{\left\{#1\right.}}
\providecommand{\rcbrak}[1]{\ensuremath{\left.#1\right\}}}
\theoremstyle{remark}
\newtheorem{rem}{Remark}
\newcommand{\sgn}{\mathop{\mathrm{sgn}}}
\providecommand{\abs}[1]{\left\vert#1\right\vert}
\providecommand{\res}[1]{\Res\displaylimits_{#1}} 
\providecommand{\norm}[1]{\left\lVert#1\right\rVert}
%\providecommand{\norm}[1]{\lVert#1\rVert}
\providecommand{\mtx}[1]{\mathbf{#1}}
\providecommand{\mean}[1]{E\left[ #1 \right]}
\providecommand{\fourier}{\overset{\mathcal{F}}{ \rightleftharpoons}}
%\providecommand{\hilbert}{\overset{\mathcal{H}}{ \rightleftharpoons}}
\providecommand{\system}{\overset{\mathcal{H}}{ \longleftrightarrow}}
	%\newcommand{\solution}[2]{\textbf{Solution:}{#1}}
\newcommand{\solution}{\noindent \textbf{Solution: }}
\newcommand{\cosec}{\,\text{cosec}\,}
\providecommand{\dec}[2]{\ensuremath{\overset{#1}{\underset{#2}{\gtrless}}}}
\newcommand{\myvec}[1]{\ensuremath{\begin{pmatrix}#1\end{pmatrix}}}
\newcommand{\mydet}[1]{\ensuremath{\begin{vmatrix}#1\end{vmatrix}}}

\numberwithin{equation}{subsection}

\makeatletter
\@addtoreset{figure}{problem}
\makeatother
\let\StandardTheFigure\thefigure
\let\vec\mathbf

\renewcommand{\thefigure}{\theproblem}

\def\putbox#1#2#3{\makebox[0in][l]{\makebox[#1][l]{}\raisebox{\baselineskip}[0in][0in]{\raisebox{#2}[0in][0in]{#3}}}}
     \def\rightbox#1{\makebox[0in][r]{#1}}
     \def\centbox#1{\makebox[0in]{#1}}
     \def\topbox#1{\raisebox{-\baselineskip}[0in][0in]{#1}}
     \def\midbox#1{\raisebox{-0.5\baselineskip}[0in][0in]{#1}}
\vspace{3cm}
\title{Assignment 11}
\author{Sri Harsha CH}

\maketitle
\newpage

\bigskip
\renewcommand{\thefigure}{\theenumi}
\renewcommand{\thetable}{\theenumi}

\begin{abstract}
This document explains the method of finding whether two system of linear equations are equivalent or not.
\end{abstract}

Download all python codes from 
\begin{lstlisting}
https://github.com/harshachinta/EE5609-Matrix-Theory/tree/master/Assignments/Assignment11/code
\end{lstlisting}
%
and latex-tikz codes from 
%
\begin{lstlisting}
https://github.com/harshachinta/EE5609-Matrix-Theory/tree/master/Assignments/Assignment11
\end{lstlisting}
%
\section{Problem}
Are the following two systems of linear equations equivalent?
\begin{equation} \label{eq:eq1}
\begin{split}
    -x_1+x_2+4x_3=0\\
    x_1+3x_2+8x_3=0\\
    \frac{1}{2}x_1+x_2+\frac{5}{2}x_3=0
\end{split}
\end{equation}

\begin{equation} \label{eq:eq2}
\begin{split}
x_1-x_3=0\\
x_2+3x_3=0
\end{split}
\end{equation}
\section{Explanation}
System of linear equations in \eqref{eq:eq1} can be expressed in matrix form as,
\begin{align}
    &\vec{A}\vec{x}=0\\
    &\myvec{-1&1&4\\1&3&8\\\frac{1}{2}&1&\frac{5}{2}}\vec{x}=0
\end{align}
System of linear equations in \eqref{eq:eq2} can be expressed in matrix form as,
\begin{align}
    &\vec{B}\vec{x}=0\\
    &\myvec{1&0&-1\\0&1&3}\vec{x}=0
\end{align}
Two system of linear equations are equivalent if one system can be expressed as a linear combination of other system.\\
Matrix $\vec{B}$ can be obtained from matrix $\vec{A}$ as,
\begin{align}
&   \vec{B} = \vec{C}\vec{A}\\
&   \myvec{1&0&-1\\0&1&3} = \vec{C}\myvec{-1&1&4\\1&3&8\\\frac{1}{2}&1&\frac{5}{2}}\\
& \vec{C} = \myvec{-1&1&-2\\\frac{1}{2}&-\frac{1}{2}&2}
\end{align}
Now, writing equations in matrix-vector form,
\begin{align*}
    &x_1-x_3=\myvec{1&0&-1}\vec{x}
\end{align*}
\begin{multline}
\implies \myvec{1&0&-1}\vec{x} = -1\myvec{-1&1&4}\vec{x}\\+1\myvec{1&3&8}\vec{x}-2\myvec{\frac{1}{2}&1&\frac{5}{2}}\vec{x}\label{eq:eq3}
\end{multline}
\begin{align*}
    &x_2+3x_3=\myvec{0&1&3}\vec{x}
\end{align*}
\begin{multline}
\implies \myvec{0&1&3}\vec{x} = \frac{1}{2}\myvec{-1&1&4}\vec{x}\\-\frac{1}{2}\myvec{1&3&8}\vec{x}+2\myvec{\frac{1}{2}&1&\frac{5}{2}}\vec{x}\label{eq:eq4}
\end{multline}
Equations \eqref{eq:eq3} and \eqref{eq:eq4} is same as multiplying $\vec{C}$ with $\vec{A}$ which is the linear combination of rows of matrix $\vec{A}$.\\
Thus each equation in second system can be expressed as linear combination of the equations in first system.\\
Therefore, the two system of linear equations are equivalent.
\section{Solution}
The two systems of linear equations are equivalent
\\\end{document}
