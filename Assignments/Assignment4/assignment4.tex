\documentclass[journal,12pt,twocolumn]{IEEEtran}

\usepackage{setspace}
\usepackage{gensymb}

\singlespacing


\usepackage[cmex10]{amsmath}

\usepackage{amsthm}

\usepackage{mathrsfs}
\usepackage{txfonts}
\usepackage{stfloats}
\usepackage{bm}
\usepackage{cite}
\usepackage{cases}
\usepackage{subfig}

\usepackage{longtable}
\usepackage{multirow}

\usepackage{enumitem}
\usepackage{mathtools}
\usepackage{steinmetz}
\usepackage{tikz}
\usepackage{circuitikz}
\usepackage{verbatim}
\usepackage{tfrupee}
\usepackage[breaklinks=true]{hyperref}

\usepackage{tkz-euclide}

\usetikzlibrary{calc,math}
\usepackage{listings}
    \usepackage{color}                                            %%
    \usepackage{array}                                            %%
    \usepackage{longtable}                                        %%
    \usepackage{calc}                                             %%
    \usepackage{multirow}                                         %%
    \usepackage{hhline}                                           %%
    \usepackage{ifthen}                                           %%
    \usepackage{lscape}     
\usepackage{multicol}
\usepackage{chngcntr}

\DeclareMathOperator*{\Res}{Res}

\renewcommand\thesection{\arabic{section}}
\renewcommand\thesubsection{\thesection.\arabic{subsection}}
\renewcommand\thesubsubsection{\thesubsection.\arabic{subsubsection}}

\renewcommand\thesectiondis{\arabic{section}}
\renewcommand\thesubsectiondis{\thesectiondis.\arabic{subsection}}
\renewcommand\thesubsubsectiondis{\thesubsectiondis.\arabic{subsubsection}}


\hyphenation{op-tical net-works semi-conduc-tor}
\def\inputGnumericTable{}                                 %%

\lstset{
%language=C,
frame=single, 
breaklines=true,
columns=fullflexible
}
\begin{document}


\newtheorem{theorem}{Theorem}[section]
\newtheorem{problem}{Problem}
\newtheorem{proposition}{Proposition}[section]
\newtheorem{lemma}{Lemma}[section]
\newtheorem{corollary}[theorem]{Corollary}
\newtheorem{example}{Example}[section]
\newtheorem{definition}[problem]{Definition}

\newcommand{\BEQA}{\begin{eqnarray}}
\newcommand{\EEQA}{\end{eqnarray}}
\newcommand{\define}{\stackrel{\triangle}{=}}
\bibliographystyle{IEEEtran}

\providecommand{\mbf}{\mathbf}
\providecommand{\pr}[1]{\ensuremath{\Pr\left(#1\right)}}
\providecommand{\qfunc}[1]{\ensuremath{Q\left(#1\right)}}
\providecommand{\sbrak}[1]{\ensuremath{{}\left[#1\right]}}
\providecommand{\lsbrak}[1]{\ensuremath{{}\left[#1\right.}}
\providecommand{\rsbrak}[1]{\ensuremath{{}\left.#1\right]}}
\providecommand{\brak}[1]{\ensuremath{\left(#1\right)}}
\providecommand{\lbrak}[1]{\ensuremath{\left(#1\right.}}
\providecommand{\rbrak}[1]{\ensuremath{\left.#1\right)}}
\providecommand{\cbrak}[1]{\ensuremath{\left\{#1\right\}}}
\providecommand{\lcbrak}[1]{\ensuremath{\left\{#1\right.}}
\providecommand{\rcbrak}[1]{\ensuremath{\left.#1\right\}}}
\theoremstyle{remark}
\newtheorem{rem}{Remark}
\newcommand{\sgn}{\mathop{\mathrm{sgn}}}
\providecommand{\abs}[1]{\left\vert#1\right\vert}
\providecommand{\res}[1]{\Res\displaylimits_{#1}} 
\providecommand{\norm}[1]{\left\lVert#1\right\rVert}
%\providecommand{\norm}[1]{\lVert#1\rVert}
\providecommand{\mtx}[1]{\mathbf{#1}}
\providecommand{\mean}[1]{E\left[ #1 \right]}
\providecommand{\fourier}{\overset{\mathcal{F}}{ \rightleftharpoons}}
%\providecommand{\hilbert}{\overset{\mathcal{H}}{ \rightleftharpoons}}
\providecommand{\system}{\overset{\mathcal{H}}{ \longleftrightarrow}}
	%\newcommand{\solution}[2]{\textbf{Solution:}{#1}}
\newcommand{\solution}{\noindent \textbf{Solution: }}
\newcommand{\cosec}{\,\text{cosec}\,}
\providecommand{\dec}[2]{\ensuremath{\overset{#1}{\underset{#2}{\gtrless}}}}
\newcommand{\myvec}[1]{\ensuremath{\begin{pmatrix}#1\end{pmatrix}}}
\newcommand{\mydet}[1]{\ensuremath{\begin{vmatrix}#1\end{vmatrix}}}

\numberwithin{equation}{subsection}

\makeatletter
\@addtoreset{figure}{problem}
\makeatother
\let\StandardTheFigure\thefigure
\let\vec\mathbf

\renewcommand{\thefigure}{\theproblem}

\def\putbox#1#2#3{\makebox[0in][l]{\makebox[#1][l]{}\raisebox{\baselineskip}[0in][0in]{\raisebox{#2}[0in][0in]{#3}}}}
     \def\rightbox#1{\makebox[0in][r]{#1}}
     \def\centbox#1{\makebox[0in]{#1}}
     \def\topbox#1{\raisebox{-\baselineskip}[0in][0in]{#1}}
     \def\midbox#1{\raisebox{-0.5\baselineskip}[0in][0in]{#1}}
\vspace{3cm}
\title{Assignment 4}
\author{Sri Harsha CH}

\maketitle
\newpage

\bigskip
\renewcommand{\thefigure}{\theenumi}
\renewcommand{\thetable}{\theenumi}

\begin{abstract}
This document explains the concept of computing the determinant of a matrix given.
\end{abstract}

Download all python codes from 
\begin{lstlisting}
https://github.com/harshachinta/EE5609-Matrix-Theory/tree/master/Assignments/Assignment4/code
\end{lstlisting}
%
and latex-tikz codes from 
%
\begin{lstlisting}
https://github.com/harshachinta/EE5609-Matrix-Theory/tree/master/Assignments/Assignment4
\end{lstlisting}
%
\section{Problem}
If $a,b$ and $c$ are real numbers, and\\
$\triangle = \mydet{b+c & c+a & a+b\\c+a & a+b & b+c\\a+b&b+c & c+a} =0$, \\Show that either $a+b+c =0$ or $a=b=c$.
\section{Explanation}
Given,\\
\begin{align}
&\triangle = \mydet{b+c & c+a & a+b\\c+a & a+b & b+c\\a+b&b+c & c+a} \label{eq:ques}\\
\xleftrightarrow[]{C_1\leftarrow C_1+C_2+C_3}
&\mydet{2(a+b+c) & c+a & a+b\\2(a+b+c) & a+b & b+c\\2(a+b+c)&b+c & c+a} \label{eq:eq_1}\\
&=2(a+b+c)\mydet{1 & c+a & a+b\\1 & a+b & b+c\\1&b+c & c+a} \label{eq:eq_2}\\
\xleftrightarrow[]{R_1\leftarrow R_1-R_2; R_2 \leftarrow R_2-R_3}
&2(a+b+c)\mydet{0 & c-b & a-c\\0 & a-c & b-a\\1&b+c & c+a} =0 \label{eq:equ_3}
\end{align}
On expanding determinant along first column from equation \eqref{eq:equ_3},
\begin{multline}
2(a+b+c)[(c-b)(b-a)-(a-c)^{2}]=0 \nonumber
\end{multline}
\begin{multline}
\implies 2(a+b+c)(a^{2}+b^{2}+c^{2}\\
-ab-bc-ca)=0 \nonumber
\end{multline}
\begin{multline}
\implies (a+b+c)(2a^{2}+2b^{2}+2c^{2}\\
-2ab-2bc-2ca)=0\nonumber
\end{multline}
\begin{multline}
\implies (a+b+c)\\
[(a-b)^{2}+(b-c)^{2}+(c-a)^{2}] = 0 \label{eq:equ_4}
\end{multline}
From equation \eqref{eq:equ_4} we get 2 equations,
\begin{align}
&\implies \boxed{(a+b+c) = 0} \label{eq:sol_1}
\end{align}
or,
\begin{align}
&\implies (a-b)^{2}+(b-c)^{2}+(c-a)^{2} = 0 \label{eq:sol_2}
\end{align}
Equation \eqref{eq:sol_2} is possible only when, $a=b=c$
\begin{align}
&\implies \boxed{a=b=c} \label{eq:sol_3}
\end{align}
From equation \eqref{eq:sol_1} and \eqref{eq:sol_3} we can say that,
$\triangle=0$ if $a+b+c=0$ or $a=b=c$.
\section{Solution}

From equation \eqref{eq:sol_1} and \eqref{eq:sol_3} we can say that,
$\triangle=0$ if $a+b+c=0$ or $a=b=c$.

\\\end{document}
